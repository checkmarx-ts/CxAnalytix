\chapter{Configuration}\label{chap:configuration}

\newcommand{\expandsenv}{\faCogs\ }
\newcommand{\encrypts}{\faLock\ }
\newcommand{\contentvariables}{\faEye\ }


\newtcblisting{xml}[3]{
    listing only,
    title=<#1> #2,
    width=\textwidth,
    listing options={
        basicstyle=\small\ttfamily,
        breaklines=true,
        columns=fullflexible,
    },
}


CxAnalytix uses the XML \texttt{cxanalytix.config} to configure the application execution parameters.  A path search order described in Chapter \ref{chap:installation}
describes how CxAnalytix finds the configuration file at runtime.

\noindent\\Configuration of CxAnalytix is performed by adding XML elements to the top level \texttt{configuration} element in the
\texttt{cxanalytix.config} file.  The XML element names are indicated in each section with an example set of XML attributes or child-elements.

\paragraph{Element Annotation Key}

\noindent\\\\Elements annotated with \expandsenv will expand environment variables for assigned to elements for environment variable values in the
format \texttt{\%VARIABLENAME\%}.  

\noindent\\Elements annotated with \contentvariables allow for expansion of record content as described in Appendix \ref{ShardKeyCookbook}.  

\noindent\\Elements annotated with \encrypts will encrypt the entire configuration element upon the first execution of CxAnalytix
\footnote{Only on Windows; Linux does not support configuration file encryption.}.  


\section{General Configuration}\label{sec:general}

\subsection{Never Do This}

The \texttt{cxanalytix.config} file needs to be modified to supply the appropriate configuration elements.  The \texttt{configSections}
element is not intended to be user configurable.  Listing \ref{lst:motouch} shows an example of the contents of this XML element;
\textbf{never change anything in this section.}

\begin{lstlisting}[caption={Part of the Configuration File to Never Change}, label={lst:motouch}, language=XML, basicstyle=\ttfamily\tiny]
<configSections>
    <section name="CxAnalytixService" 
        type="CxAnalytix.Configuration.Impls.CxAnalytixService, Configuration" />
    <section name="CxSASTCredentials" 
        type="CxAnalytix.Configuration.Impls.CxCredentials, Configuration" />
    <section name="CxSCACredentials" 
        type="CxAnalytix.Configuration.Impls.CxMultiTenantCredentials, Configuration" />
    <section name="CxSASTConnection" 
        type="CxAnalytix.Configuration.Impls.CxSASTConnection, Configuration" />
    <section name="CxSCAConnection" 
        type="CxAnalytix.XForm.ScaTransformer.Config.CxScaConnection, ScaTransformer" />
    <section name="ProjectFilterRegex" 
        type="CxAnalytix.Configuration.Impls.CxFilter, Configuration"/>
    <section name="CxAuditTrailSuppressions" 
        type="CxAnalytix.AuditTrails.Crawler.Config.CxAuditTrailSuppressions, CxAuditTrailsCrawler"/>
    <section name="CxAuditTrailRecords" 
        type="CxAnalytix.AuditTrails.Crawler.Config.CxAuditTrailRecordNameMap, CxAuditTrailsCrawler"/>
    <section name="CxDB" 
        type="CxAnalytix.CxAuditTrails.DB.Config.CxAuditDBConnection, CxAuditTrailsDB"/>
    <section name="AMQPConnection" 
        type="CxAnalytix.Out.AMQPOutput.Config.Impls.AmqpConnectionConfig, AMQPOutput"/>
    <section name="AMQPConfig" 
        type="CxAnalytix.Out.AMQPOutput.Config.Impls.AmqpConfig, AMQPOutput"/>
    <section name="CxLogOutput" 
        type="CxAnalytix.Out.Log4NetOutput.Config.Impl.LogOutputConfig, Log4NetOutput" />
    <section name="CxMongoOutput" 
        type="CxAnalytix.Out.MongoDBOutput.Config.Impl.MongoOutConfig, MongoDBOutput" />
    <section name="CxMongoConnection" 
        type="CxAnalytix.Out.MongoDBOutput.Config.Impl.MongoConnectionConfig, MongoDBOutput" />
</configSections>
\end{lstlisting}

\subsection{Checkmarx Service Connection Configuration}\label{sec:connection}

\subsubsection{Checkmarx SAST Connection Configuration}

Configuring a connection to Checkmarx SAST requires the elements \texttt{CxSASTConnection} and \texttt{CxSASTCredentials}.

\begin{xml}{CxSASTConnection}{\expandsenv}{}
<CxSASTConnection
    URL=""
    mnoURL=""
    TimeoutSeconds="" 
    ValidateCertificates="true"
    RetryLoop=""
    />
\end{xml}

\begin{table}[h]
    \caption{CxSASTConnection Attributes}        
    \begin{tabularx}{\textwidth}{cccl}
        \toprule
        \textbf{Attribute} & \textbf{Default} & \textbf{Required} & \textbf{Description}\\
        \midrule
        \texttt{URL} & N/A & Yes & \makecell[l]{The URL to the SAST server.}\\
        \midrule
        \texttt{mnoURL} & N/A & No & \makecell[l]{The URL to the Management and Orchestration\\endpoint of the SAST server.}\\
        \midrule
        \texttt{TimeoutSeconds} & 300 & No & \makecell[l]{The number of seconds to wait until an\\API operation times out.}\\
        \midrule
        \texttt{ValidateCertificates} & True & No & \makecell[l]{Validate SSL certificates for\\API endpoints.}\\
        \midrule
        \texttt{RetryLoop} & 0 & No & \makecell[l]{The number of retries for an API operation\\after the operation times out.}\\
        \bottomrule
    \end{tabularx}
\end{table}

\begin{xml}{CxSASTCredentials}{\expandsenv\encrypts}{}
<CxSASTCredentials
    Username=""
    Password=""
    />
\end{xml}
    
\begin{table}[h]
    \caption{CxSASTCredentials Attributes}        
    \begin{tabularx}{\textwidth}{cccl}
        \toprule
        \textbf{Attribute} & \textbf{Default} & \textbf{Required} & \textbf{Description}\\
        \midrule
        \texttt{Username} & N/A & Yes & \makecell[l]{A username for a SAST application account.}\\
        \midrule
        \texttt{Password} & N/A & Yes & \makecell[l]{The password for the SAST application account.}\\
        \bottomrule
    \end{tabularx}
\end{table}

\subsubsection{Checkmarx SCA Connection Configuration}
Configuring a connection to Checkmarx SCA requires the elements \texttt{CxSCAConnection} and \texttt{CxSCACredentials}.

\begin{xml}{CxSCAConnection}{\expandsenv}{}
<CxSCAConnection
    URL=""
    LoginURL=""
    TimeoutSeconds="" 
    ValidateCertificates="true"
    RetryLoop=""
    />
\end{xml}

\begin{table}[h]
    \caption{CxSCAConnection Attributes}        
    \begin{tabularx}{\textwidth}{cccl}
        \toprule
        \textbf{Attribute} & \textbf{Default} & \textbf{Required} & \textbf{Description}\\
        \midrule
        \texttt{URL} & N/A & Yes & \makecell[l]{The URL to the SCA API.}\\
        \midrule
        \texttt{LoginURL} & N/A & No & \makecell[l]{The URL to the SCA access control endpoint.}\\
        \midrule
        \texttt{TimeoutSeconds} & 300 & No & \makecell[l]{The number of seconds to wait until an\\API operation times out.}\\
        \midrule
        \texttt{ValidateCertificates} & True & No & \makecell[l]{Validate SSL certificates for\\API endpoints.}\\
        \midrule
        \texttt{RetryLoop} & 0 & No & \makecell[l]{The number of retries for an API operation\\after the operation times out.}\\
        \bottomrule
    \end{tabularx}
\end{table}

\begin{xml}{CxSCACredentials}{\expandsenv\encrypts}{}
<CxSCACredentials
    Username=""
    Password=""
    Tenant=""
    />
\end{xml}
    
\begin{table}[h]
    \caption{CxSCACredentials Attributes}        
    \begin{tabularx}{\textwidth}{cccl}
        \toprule
        \textbf{Attribute} & \textbf{Default} & \textbf{Required} & \textbf{Description}\\
        \midrule
        \texttt{Username} & N/A & Yes & \makecell[l]{A username for an SCA application account.}\\
        \midrule
        \texttt{Password} & N/A & Yes & \makecell[l]{The password for the SCA application account.}\\
        \midrule
        \texttt{Tenant} & N/A & Yes & \makecell[l]{The name of the SCA tenant.}\\
        \bottomrule
    \end{tabularx}
\end{table}


\subsection{CxAnalytix Service and CLI Execution Configuration}\label{sec:runtime_config}

The \texttt{CxAnalytixService} element provides the runtime configuration for CxAnalytix.  The child element \texttt{EnabledTransformers}
is configured with the transformation logic modules to use when crawling Checkmarx services.

\begin{xml}{CxAnalytixService}{\expandsenv}{}
<CxAnalytixService
    InstanceId=""
    ConcurrentThreads=""
    StateDataStoragePath=""
    ProcessPeriodMinutes=""
    OutputModuleName=""
    SASTScanSummaryRecordName=""
    SASTScanDetailRecordName=""
    SCAScanSummaryRecordName=""
    SCAScanDetailRecordName=""
    ProjectInfoRecordName=""
    PolicyViolationsRecordName="">

    <EnabledTransformers>
        <Transformer Name="" />
    </EnabledTransformers>

</CxAnalytixService>
\end{xml}
        
\begin{table}[h]
    \caption{CxAnalytixService Attributes}        
    \begin{tabularx}{\textwidth}{cccl}
        \toprule
        \textbf{Attribute} & \textbf{Default} & \textbf{Required} & \textbf{Description}\\
        \midrule
        \texttt{InstanceId} & N/A & No & \makecell[l]{A static value added to each data record\\
        to indicate the CxAnalytix instance\\
        from which the record originated.}\\
        \midrule
        \texttt{ConcurrentThreads} & N/A & Yes & \makecell[l]{The number of reports that are processed\\
        concurrently.  This applies per \\
        transformation module, therefore using \\
        2 threads and 2 transformation modules\\
        yields 4 concurrent threads.}\\
        \midrule
        \texttt{StateDataStoragePath} & N/A & Yes & \makecell[l]{A path to a folder where the state data\\
        that is persisted between each scan is\\stored.}\\
        \midrule
        \texttt{ProcessPeriodMinutes} & N/A & Yes & \makecell[l]{The number of minutes between
        \\performing crawls for new scan\\
        results. Ignored by CxAnalytixCLI.}\\
        \midrule
        \texttt{OutputModuleName} & N/A & Yes & \makecell[l]{The name of the output module to use\\
        for data output.  The acceptable\\
        values can be found in\\
        the \hyperref[lst:outmodules]{Available Output Modules} list.}\\
        \midrule
        \texttt{SASTScanSummaryRecordName}\\
        \texttt{SASTScanDetailRecordName}\\
        \texttt{SCAScanSummaryRecordName}\\
        \texttt{SCAScanDetailRecordName}\\
        \texttt{ProjectInfoRecordName}\\
        \texttt{PolicyViolationsRecordName} & N/A & Yes & \makecell[tl]{The name of the corresponding\\
        record collection configured\\
        in the output.}\\
        \bottomrule
    \end{tabularx}
\end{table}


\paragraph{Available Output Modules}\label{lst:outmodules}
\begin{itemize}
    \item Log4Net
    \item AMQP
    \item MongoDB
\end{itemize}

\noindent\\The child element \texttt{EnabledTransformers} defines one or more transformer modules that crawl corresponding Checkmarx services.  The
attribute \texttt{Name} of the child element \texttt{Transformer} can be given one of the following values:\\

\begin{itemize}
    \item SAST
    \item SCA
\end{itemize}

\noindent\\One or more \texttt{Transformer} elements are required.  In the example below, both the SAST and SCA transformers are configured.
The selected services must have corresponding connection configurations as described in Section \ref{sec:connection}.

\begin{xml}{CxAnalytixService> \ \faArrowLeft \ <EnabledTransformers}{}{}
<CxAnalytixService ... >
    <EnabledTransformers>
        <Transformer Name="SAST" />
        <Transformer Name="SCA" />
    </EnabledTransformers>
</CxAnalytixService>
\end{xml}

\subsection{Limiting the Scope of Crawling by Filtering}

The optional \texttt{ProjectFilterRegex} configuration element can be used to limit the scope of the data crawl to only those scans matching Team or Project
name regular expressions.  The filtering is performed using a regular expression to evaluate Team and Project path such that the values of each must
match the provided regular expression.  If this \texttt{ProjectFilterRegex} is not included in the configuration file, all scans are crawled and exported.

\noindent\\The \texttt{Team} and \texttt{Project} attributes are optional.  Omitting one of the attributes or configuring the attribute with an empty
value indicates all values match.  Negative matching regular expressions also work; the typical application of this configuration is to limit crawling
to projects that are deployed to production.

\noindent\\In the example XML, the configuration crawls scans for projects meeting the following criteria:

\begin{itemize}
    \item The team does not contain the word "AppSec" anywhere in the team path.
    \item The project name contains the word "master".
\end{itemize}

\begin{xml}{ProjectFilterRegex}{}{}
<ProjectFilterRegex 
    Team="^((?!AppSec).)*\$" 
    Project="master"
    />
\end{xml}
    


\section{Output Configuration}

CxAnalytix uses the concept of an output module for storing crawled data.  Each module is implemented such that it uses the concept of "record names" 
to store data by data type.  The name of the record used by the output module is defined in the \texttt{CxAnalytixService} configuration element described
in Section \ref{sec:runtime_config}.

\noindent\\As an example, if the \texttt{SASTScanDetailRecordName} attribute in the \texttt{CxAnalytixService} configuration element is set to
"SAST\_Scan\_Detail" and the \texttt{OutputModuleName} is set to \texttt{MongoDB}, a collection named "SAST\_Scan\_Detail" will be created in MongoDB.
All static analysis vulnerability details will therefore be written into the collection named "SAST\_Scan\_Detail".

\subsection{Log4Net Output Configuration}

The Log4Net configuration is typically limited to modifying the \texttt{cxanalytics.log4net} file to modify the output paths of the
data output files.  Modifying the output location is described in Section \ref{sec:log4net_out}.

\noindent\\The \texttt{CxLogOutput} optional configuration can be used to periodically purge the files created by CxAnalytix As
scans are crawled over time.  It is presumed that the files created are being forwarded for storage, thus can be purged periodically.

\begin{xml}{CxLogOutput}{}{}
<CxLogOutput DataRetentionDays="14" OutputRoot="logs\">
    <PurgeSpecs>
        <spec MatchSpec="*.log.*" />
    </PurgeSpecs>
</CxLogOutput>
\end{xml}
            
\begin{table}[h]
    \caption{CxLogOutput Attributes}        
    \begin{tabularx}{\textwidth}{cccl}
        \toprule
        \textbf{Attribute} & \textbf{Default} & \textbf{Required} & \textbf{Description}\\
        \midrule
        \texttt{DataRetentionDays} & N/A & Yes & \makecell[l]{The maximum number of days a file can be\\
        untouched before purge.}\\
        \midrule
        \texttt{OutputRoot} & N/A & Yes & \makecell[l]{The root folder where logs are written.}\\
        \bottomrule
    \end{tabularx}
\end{table}

\noindent\\The \texttt{PurgeSpecs} element contains one or more \texttt{spec} child elements.  The attribute \texttt{MatchSpec} in each \texttt{spec} element
defines a file mask used for selecting files to purge.  The search for files matching one or more of the specifications are performed recursively
from the root.  Any files found matching one or more of the specifications are purged if the last modified date is older than the value configured
in the \texttt{DataRetentionDays} element.




\subsection{MongoDB Output Configuration}


\subsubsection{MongoDB Connection String Configuration}

The MongoDB connection is configured with a connection string following the \href{https://docs.mongodb.com/manual/reference/connection-string/}{MongoDB URI Format}.
XML reserved characters will need to be XML escaped:

\begin{itemize}
    \item \textbf{"} \ \faArrowRight \ \texttt{\&quot;}
    \item \textbf{'} \ \faArrowRight \ \texttt{\&apos;}
    \item \textbf{<} \ \faArrowRight \ \texttt{\&lt;}
    \item \textbf{>} \ \faArrowRight \ \texttt{\&gt;}
    \item \textbf{\&} \ \faArrowRight \ \texttt{\&amp;}
\end{itemize}

\noindent\\Other special characters that are not valid in a URL may need to be \href{https://www.w3schools.com/tags/ref_urlencode.ASP}{URL Encoded}.\\

\begin{xml}{CxMongoConnection}{\expandsenv\encrypts}{}
<CxMongoConnection
    ConnectionString="mongodb://<server>:27017/<database>"
/>
\end{xml}

\subsubsection{MongoDB Shard Keys}

The \texttt{CxMongoOutput} configuration element is optional; it can be used to add an additional field to each record
added to an output record with a name matching the record name configuration described in Section \ref{sec:runtime_config}.  The fields
available for each record can be found in Appendix \ref{chap:spec}. Details about creating shard keys can be found in Appendix \ref{ShardKeyCookbook}.

\noindent\\The example \texttt{CxMongoOutput} element below shows a shard key with the name \texttt{pkey} added to documents
written to collections named \texttt{SAST\_Scan\_Summary} and \texttt{SAST\_Scan\_Detail}.

\begin{xml}{CxMongoOutput}{\contentvariables}{}
<CxMongoOutput>
    <GeneratedShardKeys>
        <Spec 
            KeyName="pkey" 
            CollectionName="SAST_Scan_Summary" 
            FormatSpec="{ScanType}-{ScanFinished:yyyy-dddd}"
            />
        <Spec 
            KeyName="pkey" 
            CollectionName="SAST_Scan_Detail" 
            FormatSpec="{ScanType}-{QueryGroup}-{ScanFinished:yyyy-dddd}" 
            NoHash="true" 
            />
    </GeneratedShardKeys>
</CxMongoOutput>
\end{xml}
    

\begin{table}[h]
    \caption{CxMongoOutput Shard Key Specification Attributes}        
    \begin{tabularx}{\textwidth}{cccl}
        \toprule
        \textbf{Attribute} & \textbf{Default} & \textbf{Required} & \textbf{Description}\\
        \midrule
        \texttt{KeyName} & N/A & Yes & \makecell[l]{The name of the field in the document where the\\shard key value is written.}\\
        \midrule
        \texttt{CollectionName} & N/A & Yes & \makecell[l]{The name of the collection where documents\\
        containing this shard key value are written.}\\
        \midrule
        \texttt{FormatSpec} & N/A & Yes & \makecell[l]{A specifier composed of static values\\
        and dynamic values extracted from the document\\
        prior to being written to the collection.}\\
        \midrule
        \texttt{NoHash} & False & No & \makecell[l]{When False, the value created by the \texttt{FormatSpec}\\
        attribute is written as a Base64 encoded hash\\
        to the field with the name defined by\\
        the \texttt{KeyName} attribute.  Otherwise, the\\
        unhashed value is written.}\\
        \bottomrule
    \end{tabularx}
\end{table}


\subsection{AMQP Output Configuration}

\section{Splunk Output Configuration}\label{sec:splunk_config}
\subsection{Log4Net Output Configuration}\label{sec:log4net_out}

The Log4Net output is used to generate log files that are tailed and forwarded to Splunk via the Splunk Universal Forwarder.  This requires the Log4Net
output to be configured so that the Universal Forwarder can find the generated output files.  Please refer to Section \ref{sec:runtime_config} for details
about choosing the Log4Net output module.

\noindent\\The Log4Net configuration file should be modified only to change the output path of the generated data output files.  The generated data output files are different
than the application logging output files in that the data output files contain data to be used for analysis purposes. It may be desirable to also forward
the application log files to Splunk for monitoring and troubleshooting purposes.

\noindent\\The \texttt{CxAnalytixService} configuration section in \texttt{cxanalytix.config} contains record name mapping attributes.  Listing \ref{lst:record_map}
shows an example configuration with the record names mapped to file logger names shown configured in listing \ref{lst:record_loggers}.  The loggers
reference file appenders, as seen in listing \ref{lst:record_appenders}.  The appender configuration, by default, places all output in the \texttt{logs}
directory, which resolves to the current working directory set when a CxAnalytix process executes.  The location of the output files can be changed
by modifying the appender configuration in \texttt{cxanalytix.log4net}.


\begin{lstlisting}[caption={Example Record Map Configuration}, label={lst:record_map}, language=XML]
<CxAnalytixService 
    ConcurrentThreads="2" 
    StateDataStoragePath="%CHECKMARX_STATE_PATH%"
    ProcessPeriodMinutes="120"
    OutputModuleName="log4net"
    SASTScanSummaryRecordName="RECORD_SAST_Scan_Summary"
    SASTScanDetailRecordName="RECORD_SAST_Scan_Detail"
    SCAScanSummaryRecordName="RECORD_SCA_Scan_Summary"
    SCAScanDetailRecordName="RECORD_SCA_Scan_Detail"
    ProjectInfoRecordName="RECORD_Project_Info"
    PolicyViolationsRecordName="RECORD_Policy_Violations">
    <EnabledTransformers>
        <Transformer Name="SAST" />
    </EnabledTransformers>
</CxAnalytixService>
\end{lstlisting}

\begin{lstlisting}[caption={Log4Net Logger Configurations}, label={lst:record_loggers}, language=XML]
.. snip ..
<logger name="RECORD_SAST_Scan_Summary" aditivity="false">
    <level value="ALL" />
    <appender-ref ref="SAST_SS" />
</logger>
  .. snip ..
\end{lstlisting}

\begin{lstlisting}[caption={Log4Net Record File Appenders}, label={lst:record_appenders}, language=XML]
.. snip ..
<appender name="SAST_SS" type="log4net.Appender.RollingFileAppender">
    <appendToFile value="true" />
    <maximumFileSize value="100MB" />
    <rollingStyle value="Composite" />
    <staticLogFileName value="false" />
    <countDirection value="1" />
    <file type="log4net.Util.PatternString" value="logs/sast_scan_summary" />
    <datePattern value="'.'yyyy_MM_dd'.log'" />
    <preserveLogFileNameExtension value="true" />

    <layout type="log4net.Layout.PatternLayout">
        <conversionPattern value="%message%newline" />
    </layout>
</appender>
.. snip ..
\end{lstlisting}

\subsection{Splunk Universal Forwarder Configuration}
The \ref{https://www.splunk.com/en_us/download/universal-forwarder.html}{Splunk Universal Forwarder} is used to send data to Splunk Enterprise or Splunk Cloud.  
Please refer to the Splunk website for information for details about installing and configuring the Universal Forwarder.


\subsubsection{Output File Tailing Configuration}

Assuming an installed forwarder is able to connect to the desired Splunk instance, 
create the \href{https://docs.splunk.com/Documentation/Splunk/latest/Admin/Inputsconf}{\texttt{inputs.conf}} file at the appropriate location 
(e.g. \texttt{/etc/apps/splunkclouduf/default/inputs.conf}).  Listing \ref{lst:inputsconf} shows an example of an \texttt{inputs.conf}
file with monitoring stanzas appropriate for each type of record. 

\begin{lstlisting}[caption={Log4Net Record File Appenders}, label={lst:inputsconf}, language=XML]
[monitor://{path to logs}\CxAnalytixService...]
sourcetype=service

[monitor://{path to logs}\sast_scan_summary...]
sourcetype=sast_scan_summary

[monitor://{path to logs}\sast_scan_detail...]
sourcetype=sast_scan_detail

[monitor://{path to logs}\sast_project_info...]
sourcetype=sast_project_info

[monitor://{path to logs}\sast_policy_violations...]
sourcetype=sast_policy_violation

[monitor://{path to logs}\sca_scan_summary...]
sourcetype=sca_scan_summary

[monitor://{path to logs}\sca_scan_detail...]
sourcetype=sca_scan_detail

[monitor://{path to logs}\CxActivity_dbo_AuditTrail...]
sourcetype=cxactivity_audittrail

[monitor://{path to logs}\CxActivity_dbo_Audit_Scans...]
sourcetype=cxactivity_auditscans

[monitor://{path to logs}\CxActivity_dbo_Audit_Reports...]
sourcetype=cxactivity_auditreports

[monitor://{path to logs}\CxActivity_dbo_Audit_Queries...]
sourcetype=cxactivity_auditqueries

[monitor://{path to logs}\CxActivity_dbo_Audit_Projects...]
sourcetype=cxactivity_auditprojects

[monitor://{path to logs}\CxActivity_dbo_Audit_Presets...]
sourcetype=cxactivity_auditpresets

[monitor://{path to logs}\CxActivity_dbo_Audit_DataRetention...]
sourcetype=cxactivity_auditdataretention
\end{lstlisting}

\subsubsection{Configuring the Source Types}

The source types on the Splunk server need to be configured to appropriately parse JSON.  
This can be done using \href{https://docs.splunk.com/Documentation/Splunk/latest/Admin/Propsconf}{\texttt{props.conf}} (only available in Splunk Enterprise) 
or through the Splunk UI. A source type should be created the matches each record output source types as defined in 
\href{https://docs.splunk.com/Documentation/Splunk/latest/Admin/Inputsconf}{\texttt{inputs.conf}}.  Listing \ref{lst:sourcetypes} shows an example
of a source type entry.  The source type configuration needs to be performed for each source type.


\begin{lstlisting}[caption={Log4Net Record File Appenders}, label={lst:sourcetypes}, language=XML]
LINE_BREAKER=([\r\n]+)
KV_MODE=json
TRUNCATE=0
SHOULD_LINEMERGE=false
\end{lstlisting}

\noindent\\\textbf{Extracting Timestamps}\\

\noindent\\The source data contains timestamp fields that can be used as the timestamp Splunk uses when indexing the data.  Without specifying how to extract the
timestamp from each source type, the timestamp will default to the timestamp when the data was indexed.  This may work for current data, but data searches 
will also return historical data that is outside of the selected search time frame.

\noindent\\For the SAST Scan Summary, SAST Scan Detail, SCA Scan Summary, and SCA Scan Detail source types, this configuration option should be added:

\noindent\\\texttt{TIME\_PREFIX=\^.*ScanFinished".+?"}

\noindent\\For the SAST Project Info source type, this configuration option should be added:

\noindent\\\texttt{TIME\_PREFIX=\^.*LastCrawlDate".+?"}

\noindent\\For the SAST Policy Violation source type, this configuration option should be added:

\noindent\\\texttt{TIME\_PREFIX=\^.*ViolationOccurredDate".+?"}

\noindent\\For the Audit Trail source type, this configuration option should be added:

\noindent\\\texttt{TIME\_PREFIX=\^.*EndTime".+?"}

\noindent\\For the Audit\_Scans, Audit\_Reports, Audit\_Queries, Audit\_Projects, Audit\_Presets, Audit\_DataRetention, this configuration option should be added:

\noindent\\\texttt{TIME\_PREFIX=\^.*TimeStamp".+?"}



\section{MongoDB Output Configuration}\label{sec:mongo_config}
\section{AMQP Output Configuration}\label{sec:amqp_config}

\section{Audit Table Crawling Configuration}


\section{Logging Configuration}

Log4Net is used to produce both application logs and logs containing exported JSON key/value pairs when using the Logging Output component. 
CxAnalytix produces logging useful for troubleshooting and/or execution tracking purposes even when not using the Log4Net output method.  The 
\texttt{cxanalytix.log4net} file is generally located in the same directory as the \texttt{cxanalytix.config} file, and will allow for
customizing the logging output location and verbosity.