\chapter{Troubleshooting}

\section{Configuration file missing}
If you get the following error stack trace emitted on the console or in logs:

\begin{code}{Error Stack Trace}{}{}
Unhandled Exception: System.TypeInitializationException: The type initializer for 
'CxAnalytix.Configuration.Config' threw an exception. 
    ---> System.IO.FileNotFoundException: Configuration file missing.
   at CxAnalytix.Configuration.Config..cctor() in 
   c:\programdata\checkmarx\CxAnalytix\Configuration\Config.cs:line 26
   --- End of inner exception stack trace ---
   at CxAnalytixCLI.Program.Main(String[] args) in 
   c:\programdata\checkmarx\CxAnalytix\CxAnalytixCLI\Program.cs:line 47
\end{code}

\noindent\\Try setting your current directory to the same directory as the CxAnalytixCLI executable e.g.

\noindent\\\texttt{cd C:\textbackslash ProgramData\textbackslash checkmarx\textbackslash CxAnalytix\textbackslash artifacts\textbackslash Release\textbackslash}


\section{Trace Web API I/O Data}

All data I/O with the web APIs can be captured by adding a logger to the log4net configuration with the following lines:

\begin{xml}{logger}{configuration for TRACE\_NETWORK}{}
<logger name="CxRestClient.IO">
    <level value="TRACE_NETWORK" />
</logger>
\end{xml}

\section{Trace Web API Operations}

\noindent\\Capturing the web API data I/O generates a large amount of logging data.  A reduced set of data showing web API requests, timings, and 
response statuses can be captured by adding the following logger to the log4net configuration:

\begin{xml}{logger}{configuration for TRACE}{}
<logger name="CxRestClient.Utility">
  <level value="TRACE" />
</logger>
\end{xml}

\section{Execution Logging Verbosity}

\noindent\\The \texttt{TRACE\_NETWORK} log level setting as described above is generally only useful in the 
\texttt{CxRestClient.IO} namespace.  It can also be applied at the root level logger to increase the
verbosity of logging to the entire application to the maximum level possible:

\begin{xml}{root}{with appenders}{}
<root>
    <level value="TRACE_NETWORK" />
    <appender-ref ref="Console" />
    <appender-ref ref="RollingFile" />
</root>
\end{xml}

\noindent\\Applying \texttt{TRACE\_NETWORK} at the root level is really not recommended.  A level of logging that increases the verbosity of execution
logging but excludes network traffic is \texttt{TRACE}:

\begin{xml}{root}{with appenders}{}
<root>
    <level value="TRACE" />
    <appender-ref ref="Console" />
    <appender-ref ref="RollingFile" />
</root>
\end{xml}

\noindent\\The \texttt{DEBUG} level of logging is less verbose than \texttt{TRACE} and should be the first choice for troubleshooting issues:

\begin{xml}{root}{with appenders}{}
<root>
    <level value="DEBUG" />
    <appender-ref ref="Console" />
    <appender-ref ref="RollingFile" />
</root>
\end{xml}
