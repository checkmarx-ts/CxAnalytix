\chapter{Shard Key Cookbook}\label{ShardKeyCookbook}


If you have come to this documentation, you may have the need to store quite a lot of data and access such data in a scalable way.  
The MongoDB documentation about \href{https://docs.mongodb.com/manual/core/sharding-shard-key/#choosing-a-shard-key}{choosing a shard key} is a good read if you are
not familiar with sharding concepts as they relate to document databases.  In the context of CxAnalytix, sharding is primarily discussed in terms of 
storage scalability.  Scalability for efficient reads (e.g. avoiding queries across multiple shards) is beyond the scope of CxAnalytix.


\section{Shard Keys and Vulnerability Data}

The difficulty in choosing a shard key from vulnerability data is that it is difficult to predict if fields have low cardinality 
(e.g. the field has very few unique values) or high cardinality (e.g. the field has mostly unique values).  Even harder still is the ability to predict
the frequency at which a field's value changes.

\noindent\\ProjectName, ProjectId, TeamName are examples of low cardinality fields within any given collection.  As scans are executed against a project, 
the names of the projects will repeat quite often.  Projects are also unlikely to change teams very often.

\noindent\\ScanId is an example of a high cardinality field.  ScanId, however, may not be an ideal selection as a shard key considering it changes
for every scan.  In an extreme scenario, imagine a data storage shard being allocated for each scan.  This may result in a system where the size of
the allocated data storage is difficult to manage.

\noindent\\Combining several fields to form a composite key can often achieve sufficient cardinality.  When choosing fields, it is important to include
fields that have a sufficient change frequency.  The required frequency may depend on the volume of scanning performed.  If the frequency of change is 
very low and scan volume very high, the shard storage space may reach maximum capacity.


\section{Generated Shard/Partition Keys}
Section \ref{sec:mongo_config} details the MongoDB configuration.  The MongoDB configuration gives the ability to optionally specify a calculated
shard key to add to each record written to a collection.  This is primarily for use with cloud-based document storage systems that dynamically
expand the storage based on a value at the root of the document.  It can be used as a generic method for automating the calculation of shard
affinity if desired, but MongoDB's shard indexing capability is more flexible than using this option.

\noindent\\Each collection of documents has different fields and cardinality considerations.  Chapter \ref{chap:spec} details the fields
available for documents written in each collection.  The fields of each data record can be used to compose the shard key via
the Key Format Specification described in Appendix \ref{chap:key_format_spec}.

\noindent\\For those that wish to experiment with the Key Format Specification, the \texttt{StringFormatExamples} project can be used
for experimentation purposes.


\section{Example Shard Key Format Specifiers}

\textbf{This section is intended to be where new examples of shard keys are documented as they are chosen in field implementations.}

\noindent\\The examples below are given for consideration of a suitable shard key.  The volume of scans in an organization should be taken into
consideration when selecting a shard key.  The given examples are likely suitable for a moderate scan volume.

\noindent\\The fields \texttt{TeamName} and \texttt{ProjectName} are common to all records and are often easy to add (either both or one)
to increase cardinality.  Date fields are also generally a good choice for increasing cardinality; using the field format specifier, the cardinality increases as the
time span length decreases (e.g. year > month > day-of-week > day-of-month and so on).


\subsection{SAST Scan Summary Example}

This example uses the scan type, the year and full name of the day of the week when the scan finished.

\begin{code}{Shard Key Example for SAST\_Scan\_Summary}{}{}
<add KeyName="pkey" CollectionName="SAST_Scan_Summary"
    FormatSpec="{ScanType}-{ScanFinished:yyyy-dddd}" NoHash="true" />
\end{code}

\subsection{SAST Vulnerability Details}

This example uses the scan type, the query group, the year and full name of the day of the week when the scan finished.

\begin{code}{Shard Key Example for SAST\_Scan\_Detail}{}{}
<add KeyName="pkey" CollectionName="SAST_Scan_Detail"
    FormatSpec="{ScanType}-{QueryGroup}-{ScanFinished:yyyy-dddd}" NoHash="true" />
\end{code}
