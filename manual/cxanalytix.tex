\documentclass[a4paper, 11pt, oneside]{book}
\usepackage[a4paper, total={6.5in, 10in}]{geometry}
\usepackage{hyperref}

\usepackage[svgnames]{xcolor}
\usepackage{graphicx}
\usepackage[utf8]{inputenc}
\usepackage[T1]{fontenc}
\usepackage{PTSerif}
\usepackage{listings}
\usepackage{booktabs}
\usepackage{fouriernc}
\usepackage{makecell}
\usepackage{fontawesome}
\usepackage{hyperref}
\usepackage{tabularx}
\usepackage{xparse}
\usepackage{amssymb, amsmath}
\usepackage[many]{tcolorbox}
\tcbuselibrary{listings}


\begin{document}

\begin{titlepage}
    \thispagestyle{empty}
    \centering
    \includegraphics[scale=.4]{graphics/cx_logo-dark.png}
    \vfill
    \textcolor{Sienna}{\Huge CxAnalytix 2.x}
    \vfill
    {\Large\textbf{Nathan Leach, CSSLP\\Checkmarx Principal Solution Architect}}
\end{titlepage}

\newpage


\tableofcontents

\chapter*{Foreward}

As vulnerability scans are performed, a significant amount of data is generated over time.  The
scan data is normally consumed by development teams to remediate issues found in their specific 
projects.  The development team will perform triage and code changes as they assess what is required 
from the scan results. 

CxAnalytix was created when managment teams expressed the desire to create dashboards with
aggregated views of vulnerabilities across all scanning activity.  While CxAnalytix itself
does not produce these dashboards, it crawls scan results and collects the vulnerability
data that is then integrated into custom dashboards.\\\\

CxAnalytix currently supports the following Checkmarx products:

\begin{itemize}
    \item Checkmarx SAST (including OSA and Management \& Orchestration)
    \item Checkmarx SCA
\end{itemize}

The documentation for CxAnalytix was moved from the GitHub Wiki as of version 2.0.  This
manual will serve as the source of CxAnalytix documentation.  The manual will be updated
and versioned with each CxAnalytix release.

\newtcblisting{xml}[3]{
    listing only,
    title=<#1> #2 #3,
    width=\textwidth,
    listing options={
        basicstyle=\small\ttfamily,
        breaklines=true,
        columns=fullflexible,
    },
}

\newtcblisting{code}[3]{
    listing only,
    title=#1 #2 #3,
    width=\textwidth,
    listing options={
        basicstyle=\small\ttfamily,
        breaklines=true,
        columns=fullflexible,
    },
}


\part{Operation}
\chapter{Quickstart}

CxAnalytix performs read-only operations using Checkmarx REST APIs to obtain vulnerability data. It is therefore possible to execute 
CxAnalytix locally (e.g. on your workstation\footnote{Depending on the number of scans that need to be crawled, you may need a significant amount of disk space.}) 
or on a test system without the need to make any production changes.\\

\noindent The requirements to execute a test crawl are:

\begin{enumerate}
    \item Download the \href{https://github.com/checkmarx-ts/CxAnalytix/releases}{latest release zip}
    \item Unzip the release binaries into a directory of your choice
    \item Update the configuration with:
    \begin{itemize}
        \item The URL of the REST API endpoint(s)
        \item Credentials to access the REST API endpoint(s)
    \end{itemize}    
\end{enumerate}


\section{Quick SAST Crawl}


Some basic configuration is required before execution.

\begin{enumerate}
    \item Open a command prompt.
    \item Switch to the directory containing the CxAnalytix binaries.
    \item Set the environment variable \verb|CHECKMARX_URL| to the URL of your SAST instance.\footnote{Do not include "cxwebclient" in your SAST URL!}
    \item Set the environment variable \verb|CHECKMARX_USERNAME| to a SAST application user account name.
    \item Set the environment variable \verb|CHECKMARX_PASSWORD| to the password for the SAST application user account.
    \item Set the environment variable \verb|CHECKMARX_STATE_PATH| to a path where the state data will be stored.  
    A path of \verb|.| or any other temporary storage location is sufficient for evaluation.
    \item Execute \verb|CxAnalytixCLI| (Linux) or \verb|CxAnalytixCLI.exe| (Windows)
\end{enumerate}

The SAST vulnerability data will be retrieved and written to files in the \verb|logs| directory located in the directory where the CxAnalytix binaries
have been extracted.  If the crawl is interrupted, it will be re-started from the scan being processed when the crawl was interrupted.  To completely 
re-start the crawl, find and delete the \verb|CxAnalytixExportState.json| file in the directory specified by the \verb|CHECKMARX_STATE_PATH| 
environment variable.

\chapter{Deployment Guide}


\section{Hardware Requirements}

CxAnalytix is recommended to run on a standalone server or as a Docker container. It can run on the CxSAST manager 
as it does not tend to have an extremely CPU intensive workload for most scan volumes. It is not recommended to run it on the CxSAST manager 
if it can be avoided.

\noindent\\
Basic runtime specification:
    \begin{itemize}
        \item 8GB RAM
        \item CPU Cores should be 1 more than the configured number of concurrent scan threads (i.e. \textit{number of threads} + 1)
        \item At least 1TB of space if storing extraction data for 2 weeks or less if you expect < 100 scans every 2 hours
    \end{itemize}

\noindent\\CxAnalytix can run on any version of Linux or Windows that supports .Net 6.0.  The CxAnalytix distribution is built
as a standalone executable; this means that it is not required to install .Net 6.0 on the platform.  For some Linux distributions,
additional packages may need to be installed that are dependencies of .Net 6.0 before CxAnalytix will execute.

\section{MongoDB Sizing}

The MongoDB instance can be one of the following types of MongoDB:

\begin{itemize}
    \item Native MongoDB (local or cloud, single instance or cluster)
    \item AWS DocumentDB
\end{itemize}

\noindent Other types of MongoDB API compatible document databases may also work.  Azure's CosmoDB is currently not recommended at this time due to the 
maximum 2MB document size.  When CosmoDB officially supports the 16MB document size, it may be a viable option.

\noindent\\If persisting the extracted data in MongoDB, there is no set guide to sizing. Several variables should be considered in selection of system sizing.

\begin{itemize}
    \item Data Retention\\
    Most deployments intend to persist the vulnerability data in perpetuity (or don't have a plan to purge data). 
    If this is the case, a MongoDB cluster with the ability to expand storage should be considered. This storage expansion will likely need to consider 
    sharding configurations that can be done as part of the MongoDB server configuration or via embedding a shard key directly in the stored data. 
    The \nameref{ShardKeyCookbook} can assist with choosing an appropriate configuration.

    \item Shard Searching\\
    Most searches for vulnerability data may not be done such that a specific set of shards can limit the scope of where the query is executed. This may result
    in searches across all shards. The expected volume of queries that may result in multi-shard searches should be taken into account when specifying the 
    cluster members' hardware specifics. The machine specification will also need to account for CPU required for any indexing and data ingestion.
\end{itemize}

\subsection{Bring Your Own Indexing}

The MongoDB implementation requires that the user account configured for CxAnalytix to interact with MongoDB has read/write access to tables and 
collections specified in the configuration.  Upon the first run, tables, collections, and some default indexes will be created.

\noindent\\If there are additional indexes desired, it is possible to script your own indexes.  This includes indexes with a primary use of calculating a
destination shard at the time of write.  Each index will consume additional disk space, so the number of indexes and the fields used to define the index
should be considered when sizing your MongoDB deployment.


\noindent\\The mode of storage in CxAnalytix is intended more as a historical data store than a high-performance, transactional
document retrieval system.  This means that while it is possible to craft a shard index used to avoid crawling multiple shards, 
data organization for high performance IO is outside of the scope of CxAnalytix.


\section{Network Connectivity Requirements}

\subsection{SAST Connections Diagram}
Figure \ref{fig:SAST-network} shows the connection diagram for a CxAnalytix deployment. Where connections to REST APIs are indicated, the transport mechanism is 
HTTP/s over any port specified by the URL. Where connections to the SQL database are indicated, the connection is a regular socket using SQL server's wire protocol.

\noindent\\It is not recommended that the SQL server connection be exposed to the public internet. The SQL connection exists to support the audit log export capability 
and is not required to run CxAnalytix.\footnote{SQL connectivity is not available for Checkmarx hosted environments.}

\begin{figure}[h]
    \caption{SAST Network Connection Diagram}
    \includegraphics[width=\textwidth]{graphics/Deployment-SAST-connections.png}
    \label{fig:SAST-network}
\end{figure}

\subsection{Splunk Deployment Diagram}
Figure \ref{fig:SPLUNK-network} shows a typical deployment of CxAnalytix with the Splunk Universal Forwarder. In this deployment,
the Universal Forwarder is configured to read the CxAnalytix log output and forward the data to a remote Splunk instance. The vulnerability 
data log locations are configured in the Log4Net configuration.

\noindent\\The Splunk Universal Forwarder should be configured to tail the CxAnalytix output, which will forward to Splunk.

\begin{figure}[h]
    \caption{Splunk Network Connection Diagram}
    \includegraphics[width=\textwidth]{graphics/Deployment-SPLUNK-connections.png}
    \label{fig:SPLUNK-network}
\end{figure}


\subsection{MongoDB Deployment Diagram}
Figure \ref{fig:MONGO-network} shows a typical deployment of CxAnalytix configured to write vulnerability data to a MongoDB database. 
The MongoDB instance may be an on-premise instance or an instance hosted in the cloud.

\begin{figure}[h]
    \caption{MongoDB Network Connection Diagram}
    \includegraphics[width=\textwidth]{graphics/Deployment-MONGO-connections.png}
    \label{fig:MONGO-network}
\end{figure}


\section{Configuration Backups}
The crawl state storage files should be archived periodically to ensure crawling does not re-crawl existing scans. The state files are written to the path 
specified by the \verb|StateDataStoragePath| attribute in the \verb|CxAnalytixService| configuration section.  All files written to this path should
be archived periodically.

\noindent\\If running CxAnalytix as a Docker container, the default location \verb|/var/cxanalytix| should be mapped to a volume so that the state files are persistent across 
Docker container executions.  

\noindent\\The \verb|cxanalytix.config| file may also contain configurations that would be difficult to recreate.  It is advised that the configuration file be archived
after each change.

\section{Application Service Account}

An application service account is required for each service that CxAnalytix is configured to crawl.  If there are log messages indicating 
\verb|403: Forbidden| service API methods, this usually indicates the CxAnalytix role does not have appropriate privileges for that service.


\subsection{SAST}
CxAnalytix requires a SAST service account to authenticate with the SAST APIs to crawl scans. The service account has the following requirements:

\begin{itemize}
    \item The service account should be assigned at a team level that allows visibility to all projects that require crawling. 
    Usually this is the \verb|/CxServer|
    team but will depend on your configured team heirarchy. Any projects assigned to teams above or at a sibling level of the service account's assigned team 
    will not be visible to crawling requests.

    \item A role named CxAnalytix should be created and assigned to the service account user. The role should have the following minimum permissions:
    \begin{itemize}
        \item SAST->Project \& Scans->Save Sast Scan
        \item Reports->Generate Scan Reports
    \end{itemize}

\end{itemize}

\subsection{SCA}

CxAnalytix requires an SCA service account authenticate with the SCA APIs to crawl scan reports. The service account has the following requirements:

\begin{itemize}
    \item The service account should be assigned at a team level that allows visibility to all projects that require crawling. 
    Usually this is the \verb|/CxServer|
    team but will depend on your configured team heirarchy. Any projects assigned to teams above or at a sibling level of the service account's assigned team 
    will not be visible to crawling requests.

    \item A role named CxAnalytix should be created and assigned to the service account user. The role should have the following minimum permissions:
    \begin{itemize}
        \item SCA Viewer
    \end{itemize}

\end{itemize}


\chapter{Installation}


\section{Installing on Windows}
\section{Installing on Linux}
\section{Running CxAnalytix as a Docker Container}

CxAnalytix is published as a docker image in the \href{https://github.com/checkmarx-ts/CxAnalytix/pkgs/container/cxanalytix%2Fcxanalytix}{Checkmarx TS} GitHub 
packages repository. You can reference the image using \verb|ghcr.io/checkmarx-ts/cxanalytix/cxanalytix:<tag>| where \verb|tag| can be:

\begin{itemize}
    \item \verb|latest| to get the latest release
    \item \verb|vx.x.x| to get a specific release version
    \item \verb|vx.x.x-x-prerelease| to get a specific build of a pre-release version
\end{itemize}

\subsection{Configuration with Environment Variables}

Several of the configuration fields support environment variable expansion.  The default configuration file will use the log4net output
module to place the data files in \verb|/var/logs/cxanalytix| table \ref{tab:env} shows the environment variables required for a 
basic CxAnalytix runtime configuration.

\begin{table}
    \centering
    \begin{tabular}{|c|c|l|}
        \toprule
        \textbf{Environment Variable} & \textbf{Default} & \textbf{Description}\\
        \midrule
        \verb|CHECKMARX_URL| & None & The URL to the SAST server.\\
        \midrule
        \verb|CHECKMARX_USERNAME| & None & The username of the account that will log into CxSAST.\\
        \midrule
        \verb|CHECKMARX_PASSWORD| & None & The password for the user account logging into CxSAST.\\
        \midrule
        \verb|CHECKMARX_STATE_PATH| & \verb|/var/cxanalytix| & The path where state files will be stored.\\
        \bottomrule
    \end{tabular}
    \caption{Docker configuration environment variables}
    \label{tab:env}
\end{table}
    

\chapter{Configuration}\label{chap:configuration}


\section{General Configuration}
\section{Splunk Output Configuration}
\section{MongoDB Output Configuration}
\section{AMQP Output Configuration}


\part{Development \& Data Analysis}
\chapter{Developing for CxAnalytix}

\section{Build From Source}

\subsection{The Easy Way}
The easiest method of building the code from source is to open the \texttt{CxAnalytix.sln} file using Visual Studio (not Visual Studio Code).
From there, the entire solution can be built with a "Build Solution" command.

\subsection{From the Command Line}
If the .Net Core SDK of the appropriate version is installed, the \texttt{dotnet} command can be executed to create a build:\\

\begin{code}{Build Commands}{}{}
dotnet publish .\CxAnalytix.sln -o $OutLoc/win-x64 -c ReleaseWindows -r win-x64
dotnet publish .\CxAnalytix.sln -o $OutLoc/linux-x64 -c ReleaseLinux -r linux-x64
\end{code}

\subsection{Visual Studio Code}
This is a .Net Core application and therefore can be developed on any platform that supports .Net Core.  Readers are encouraged to investigate configuring
their instance of VSCode to build and debug CxAnalytix.  A basic \texttt{launch.json} and \texttt{tasks.json} file is included in the source tree, but they may need
updated as versions of .Net Core and VSCode change over time.


\subsection{GitHub Workflows}

The GitHub workflows are part of the official CxAnalytix repository that are used to build and publish release and pre-release builds.  Forks of the CxAnalytix
repository can also use these workflows to build and publish release artifacts.\footnote{GitHub will, by default, disable the workflows on forks.  You will need token
visit the \texttt{Actions} tab of your repository to enable the workflows.}

\noindent\\By default, the \texttt{build-ci} workflow executes on pushes, pull requests, or pull request updates on any branch.  The CI build is meant to
assure that code changes have not broken the build process.  No artifacts are published as part of the CI build.

\noindent\\The \texttt{build-release} and \texttt{build-prerelease} workflows are used to publish release and pre-release artifacts.  Platform specific zip and pdf 
artifacts are published in the code Releases section of GitHub.  A Docker image is also published in the owning organization's GitHub packages repository as 
\\\texttt{cxanalytix/cxanalytix:\textit{<tag>}}.  Each workflow requires the following Action secrets to be defined:

\begin{table}[h]
    \caption{GitHub Action Secrets}        
    \begin{tabularx}{\textwidth}{ll}
        \toprule
        \textbf{Secret Name} & \textbf{Description} \\
        \midrule
        \texttt{PACKAGE\_PAT} & \makecell[l]{A personal access token that has the appropriate permissions\\to perform build operations.} \\
        \midrule
        \texttt{PACKAGE\_USER} & The user associated with the PAT. \\
        \bottomrule
    \end{tabularx}
\end{table}

\noindent\\The package PAT requires the permissions that allow the following build operations:\\

\begin{enumerate}
    \item Set and delete tags on the CxAnalytix repository.
    \item Publish releases for the CxAnalytix repository.
    \item Publish packages in the CxAnalytix repository's owning organization.
\end{enumerate}

\subsubsection{Invoking the GitHub Release and Pre-Release Workflows}

\noindent\\The \texttt{build-release} and \texttt{build-prerelease} workflows are typically manually invoked by navigating to \texttt{Actions}
and highlighting the workflow that is to be invoked.  The \texttt{Run Workflow} button on the right side of the screen, as shown in the screen
shot below, allows a branch to be chosen and a release version to be defined.  If everything executes correctly, a release or pre-release
is published and the repository is tagged.

\noindent\\\includegraphics[scale=.4]{graphics/github-workflow.png}

\section{Developing New Modules}

The transformation and output modules are dynamically loaded at runtime.  They are developed with an SDK component included in the CxAnalytix source tree.
The SDK component is intended to be used as a project reference at this time as it is not deployed as a Nuget package.

\noindent\\The modules, being that they are dynamically loaded, will have no incoming dependencies.  This means they will not be included
in the build output since the compiler will believe the shared libraries where the modules are implemented are not in use.  For easy debug and build
purposes, each output and transform module is added as a project dependency to the \texttt{Executive} project.  The \texttt{Executive} project
is loaded by executing applications on all platforms to perform the appropriate type of CxAnalytix execution.


\subsection{Creating New Transformation Modules}

Creating a new transformer type can be done with the following steps:

\begin{enumerate}
    \item Create a class library project.
    \item Add a project reference to \texttt{SDK.Modules}
    \item Create a class that derives from \texttt{TransformerModule}
    \item Create a public default constructor that initializes the transformer module with the runtime specifics of your module
    \item Implement the abstract methods/properties defined in \texttt{TransformerModule}
\end{enumerate}

\noindent\\An example transformer module can be observed in the code snippet \hyperref[lst:xform]{Example Transformer Module}. The arguments passed to 
the \texttt{TransformerModule} are explained in table \ref{tab:xform}.\\


\begin{table}
    \centering
    \begin{tabular}{|c|c|l|}
        \toprule
        \textbf{Argument} & \textbf{Type} & \textbf{Description}\\
        \midrule
        \texttt{moduleName} & \texttt{String} & \makecell[l]{The string used to select the module for invocation in the\\
        configuration as described in section \ref{sec:general} (e.g. SAST, SCA)}\\
        \midrule
        \texttt{moduleImplType} & \texttt{Type} & The .Net type that contains the implementation of the module.\\
        \midrule
        \texttt{stateStorageFileName} & \texttt{String} & The name of the file that stores any state of the transformation module.\\
        \bottomrule
    \end{tabular}
    \caption{\texttt{TransformerModule} Construction Parameters}
    \label{tab:xform}
\end{table}



\begin{code}{Example Transformer Module}{\label{lst:xform}}{}
class Transformer : TransformerModule
{
    public override string DisplayName 
        => "This is my module, this text will show in logs";

    
    public Transformer() 
        : base("MYMODULE", typeof(Transformer), "MYMODULE_state.json")
    {
        // Initialization goes here, if any
    }

    public override void DoTransform(CancellationToken token)
    {
        // This is where the transformation is executed.
    }
}
\end{code}

\subsection{Creating New Output Modules}

Output modules are created similarly to transformer modules.  The basic tasks to create an output module:

\begin{enumerate}
    \item Create a class library project.
    \item Add a project reference to \texttt{SDK.Modules}
    \item Create a class that derives from \texttt{OutputModule}; this is your factory class that creates the instance of the output implementation
    \item Create a public default constructor that initializes the output module with the runtime specifics of your module
    \item Implement the abstract methods/properties defined in \texttt{OutputModule}
\end{enumerate}

\noindent\\The snippet \hyperref[lst:output]{Example Output Module} shows an example of an output implementation.  Table \ref{tab:output} explains the output
module arguments.


\begin{code}{Example Output Module}{\label{lst:output}}{}
public class MyOutFactory : SDK.Modules.Output.OutputModule
{
    public MyOutFactory() : base("MyOutput", typeof(MyOutFactory) )
    {
    }

    public override IRecordRef RegisterRecord(string recordName)
    {
        // Implementation for registering a record
    }

    public override IOutputTransaction StartTransaction()
    {
        // This is where the instance of the output implementation is given
        // to the caller.  Transactional capabilities are optional, the
        // concept of a transaction is simply a grouping of related
        // data being output.
    }
}    
\end{code}


\begin{table}
    \centering
    \begin{tabular}{|c|c|l|}
        \toprule
        \textbf{Argument} & \textbf{Type} & \textbf{Description}\\
        \midrule
        \texttt{moduleName} & \texttt{String} & \makecell[l]{The string used to select the module for output in the\\
        configuration as described in section \ref{sec:general} (e.g. MongoDB, AMQP, Log4Net)}\\
        \midrule
        \texttt{moduleImplType} & \texttt{Type} & The .Net type that contains the implementation of the module.\\
        \bottomrule
    \end{tabular}
    \caption{\texttt{OutputModule} Construction Parameters}
    \label{tab:output}
\end{table}


\section{Module Configuration}

\subsection{CxAnalytix Internal Configuration}\label{sec:internal_config}

It is possible to obtain the general configuration objects (described in section \ref{sec:general}) with a project dependency for the \texttt{Configuration}
class library.  Obtaining a reference to the \texttt{CxAnalytixService} configuration can then be achieved with the example code show in the 
\hyperref[lst:global_config]{code snippet below}.  Instances of CxAnalytix configuration objects can be obtained by specifying the type in the generic parameter for the 
method call \texttt{CxAnalytix.Configuration.Impls.Config.GetConfig<T>()} 


\begin{code}{Retrieving a CxAnalytixService Config Object Instance}{\label{lst:global_config}}{}
private CxAnalytixService Service => 
    CxAnalytix.Configuration.Impls.Config.GetConfig<CxAnalytixService>();
\end{code}


\subsection{Module-Specific Configuration}

All configurations in the \texttt{cxanalytix.config} file are consumed by the Microsoft\\ 
\texttt{System.Configuration.ConfigurationManager} package.
Each individual module implementation should define configuration classes that are publicly accessible for dynamic loading.
The assembly path for custom module configuration is then placed in the \texttt{configSections} element of the \texttt{cxanalytix.config}
file.  The \hyperref[lst:config_sections]{AMQP Configuration Section Entries} snippet shows an example of a configuration section entry to load the AMQP configuration elements
from the AMQP class library.

\noindent\\Obtaining an instance of a configuration object uses the same \texttt{GetConfig} as was used in section \ref{sec:internal_config}
to obtain an instance of the \texttt{CxAnalytixService} configuration object.  The \hyperref[lst:amqp_config]{Retrieving a Config Object Instance} snippet 
shows an example of how the AMQP output module obtains an instance of an module-specific configuration object.\\

\begin{code}{AMQP Configuration Section Entries}{\label{lst:config_sections}}{}
<configSections>
    <section name="AMQPConnection" 
        type="CxAnalytix.Out.AMQPOutput.Config.Impls.AmqpConnectionConfig, AMQPOutput"
        />
    <section name="AMQPConfig" 
        type="CxAnalytix.Out.AMQPOutput.Config.Impls.AmqpConfig, AMQPOutput"
        />
</configSections>
\end{code}


\begin{code}{Retrieving a Config Object Instance}{\label{lst:amqp_config}}{}
private AmqpConnectionConfig ConConfig =>
    CxAnalytix.Configuration.Impls.Config.GetConfig<AmqpConnectionConfig>();
\end{code}
    
\chapter{Analyzing CxAnalytix Data}

CxAnalytix outputs data related to vulnerabilities as they are detected and remediated over time.  This means the data collected can be classified as 
\href{https://en.wikipedia.org/wiki/Time_series}{time-series data}.  Time series data, as a basic definition, is periodically recording values that change over time.  
Each recorded value can be referred to as a "sample".

\noindent\\This generally causes some confusion as most people are accustomed to analyzing business data that essentially only records the current state of 
the business (e.g. "Give me a A/R report showing me a list of customers that have outstanding balances, grouping the past-due amounts in 30/60/90 day buckets.")  
Most of this data is organized in a relational form that is selected by understanding the relational linkage between multiple entities.  The pre-requisite for
extracting meaning from data organized relationally would be to understand which entities are related.

\noindent\\The CxAnalytix data is generally "flat" output (with a few exceptions); this means there is no knowledge required to understand the relationship between entities.
Each record (or "sample") has all values required for the context of the record without needing to understand any relationships between entities.  The technique for
deriving meaning from the data is to understand the filtering criteria needed to reduce the data set to show only the data needed for analysis.  Often this filtering is
performed as a pipeline starting with the widest filtering criteria sending data through progressively narrower filtering criteria.

\subsection{Understanding Sampling Behavior}

Performing analysis on vulnerability data requires a bit of knowledge about the circumstances by which data arrives.  Most time series data is collected from 
sensors that are emitting samples on a somewhat predictable periodic basis; vulnerability data is not collected in the same manner. Vulnerability scans are not 
necessarily performed with a predictable cadence.  There are some reasons for this:


\begin{itemize}
    \item Scanning code on commit to a repository would require commits to a repository; developers don't always work on code in a particular repository, 
    and they certainly do not have a regular pattern for committing code.
    \item Scheduled scans may fail or be delayed by lack of scanning resources.
    \item Ad-hoc scans may be interleaved between scans scheduled on a regular cadence.
    \item Code that is not under active development may not get scanned regularly.
\end{itemize}


\subsection{Factors Influencing Result Changes}

There are several variables that affect how vulnerabilities can change over time.  The most obvious one is that vulnerabilities appear and disappear based on code
changes as development is performed over time.  If this were the only factor that caused changes in detected vulnerabilities, analysis would be easy. Consider:

\begin{itemize}
    \item Upgrades to the SAST software can increase the coverage of languages, frameworks, APIs, and even coding techniques.  Vulnerability count trends may
    reflect the level of scan accuracy that product changes introduce in the upgrade.
    \item Tuning of the Preset, Exclusions, and/or CxQL query can change what is detected in a scan.
    \item The submitted code can be changed to affect the scan results.
    \item In some integration scenarios, it is possible for developers to submit code to exclude files containing vulnerable code. The issue will appear
    to have been remediated due to a change in the build process that can not be observed by static code analysis.
    \item Similarly, it is possible to inadvertently submit code that should not be scanned thus increasing the number of results.
    \item Errors in the management of the code may cause vulnerabilities that were previously fixed to reappear.
    \item Incremental scan results will likely differ significantly from full scan results.
\end{itemize}


\subsection{Identifying SAST vulnerabilities over Multiple Scans}

The SAST Vulnerability Detail records have every node for every path in the reported vulnerabilities.  Often filtering the data where \texttt{NodeId == 1}
is sufficient to reduce the amount of data that needs to be considered.  Uniquely identifying a vulnerability can be done by calculating a
"fingerprint" for the vulnerability.

\noindent\\Most approaches make the wrong assumption that \texttt{SimilarityId} can be used to identify each vulnerability across scans.  This does not work due to:

\begin{itemize}
    \item Vulnerability paths for files that are copied to different paths will have the same \texttt{SimilarityId}.
    \item Vulnerability paths for files that are scanned in multiple projects will have the same \texttt{SimilarityId}.
    \item Code that is copy/pasted multiple times in the same file may have the same \texttt{SimilarityId}.
    \item Different vulnerabilities with the same start and end node in the data flow path will have the same \texttt{SimilarityId}.
\end{itemize}


\noindent\\Identifying a specific vulnerability across scans can be done by hashing a compound identifier generated from fields in the data record. To understand 
which components to select for the compound identifier, some explanation of how record data elements can be used to derive a fingerprint is required.

\subsubsection{Project Identification}

Scans are executed under the context of a SAST project; in most cases, this SAST project represents a collection of scans over the lifetime of code evolution
for code in a single repository. \texttt{ProjectId} would therefore be a unique, non-changing value suitable for establishing a logical collection
of related scan data.

\noindent\\Another option for identifying a logical collection of related scan data would be to concetenate \texttt{TeamName} and \texttt{ProjectName} as a path.  
For example, "/CxServer/US/FinancialServices/ShoppingCart\_master" has a team name of "/CxServer/US/FinancialServices" and a project name of "ShoppingCart\_master".
This is roughly equivalent to the uniqueness of \texttt{ProjectId}, with the caveat that projects can be re-assigned to a different team.  If a project is assigned
to a different team, it is logically no longer the same project for the purposes of tracking scans over the lifetime of code scanned in the SAST project.  If your
analysis needs are to track vulnerabilities on a team level, using \texttt{TeamName} and \texttt{ProjectName} for a unique identifier may be better suited as
a way to identify a logical grouping of scans.

\noindent\\It is important to note that SAST treats vulnerabilities with the same \texttt{SimilarityId} as a single vulnerability across all projects in the same team.  
Setting a vulnerability with the status of \textbf{Not Exploitable} in one project, for example, would result in the vulnerability being marked 
as \textbf{Not Exploitable} if the same file (or a copy of it) were scanned in another project on the same team.


\subsubsection{Vulnerability Classification}

Since vulnerabilities may have the same start and end node, the \texttt{SimilarityId} value may appear under multiple vulnerability categories 
(or even multiple times per category).  The category roughly corresponds to the \texttt{QueryName}.  Often, the use of \texttt{QueryName} as a component in a 
compound identifier would be sufficient for classification since most queries won't report results for the same \texttt{QueryName} that appear in a 
different \texttt{QueryGroup} with the same \texttt{SimilarityId}.  This is usually the case given the result path is limited to a single language in all nodes of the
data flow path.

\noindent\\It is possible, however, to have language agnostic results (such as those generated by TruffleHog) that give the same result for the same 
\texttt{QueryName} under each \texttt{QueryGroup}.  Using both the \texttt{QueryGroup} and \texttt{QueryName} as part of the compound identifier would increase the 
uniqueness accuracy.

\subsubsection{Aggregation of Data from Multiple SAST Instances}

If you have multiple SAST instances and are aggregating the data into a single store, consider using \texttt{InstanceId} to compose the vulnerability fingerprint.  Data
from multiple SAST instances implies that \texttt{ProjectId} is no longer unique.

\subsubsection{Examples of Vulnerability Fingerprints}\label{sec:fingerprint}
It is important to note that code changes over time but may remain vulnerable to the same detected vulnerability.  Code format and composition affect
the ability to uniquely identify a vulnerability algorithmically.  A human may look at code changes and understand that it is the same code, but identifying
it is the same code algorithmically is significantly more complicated.  Identifying exact uniqueness of a particular vulnerability requires virtually no code changes
over time; this is not a realistic expectation for code under active development.  Code that is actively developed, however, is generally changed in small 
increments between scans.  The fingerprinting methods described here are intended to generate a unique identifier that is sufficiently unique for the purpose of 
tracing the lifecycle of a specific vulnerability.

\noindent\\It is important to note that using the fingerprint as a method for counting the number of vulnerabilities may result in differences between
the number of reported vulnerabilities in SAST scan summary views and project dashboards.  This is due to the occasional coalescing of duplicate vulnerabilities
into a single fingerprinted vulnerability.  This is generally seen when multiple vulnerabilities are reported for code that has been duplicated in multiple
locations in the scanned code.


\textbf{\noindent\\\\Fingerprint Composition: \texttt{ProjectId} + \texttt{QueryName} + \texttt{SinkFileName} + \texttt{SinkLine} + \texttt{SimilarityId}}

\noindent\\This fingerprint is generally suitable for tracking a vulnerability across multiple scans in a SAST project.    

\noindent\\For greater accuracy in tracking, consider adding the following fields:

\begin{itemize}
    \item \texttt{QueryGroup}
    \item \texttt{QueryLanguage} + \texttt{QuerySeverity}
    \item \texttt{QueryLanguage} + \texttt{ResultSeverity}
\end{itemize}

\textbf{\noindent\\\\Fingerprint Composition: \texttt{TeamName} + \texttt{QueryName} + \texttt{SinkFileName} + \texttt{SinkLine} + \texttt{SimilarityId}}

\noindent\\Using \texttt{TeamName} in place of \texttt{ProjectId} will effectively allow vulnerabilities to be assessed once for all projects assigned to a team.  There 
are some potential drawbacks to this approach:

\begin{itemize}
    \item The same code in unrelated projects may be counted as one vulnerability for all projects in the team.
    \item Projects can be moved to different teams.  Moving a project to a new team will change the timeline for the vulnerability given
    historical samples will reflect the team name at the time the sample was recorded.
    \item It may not be possible to determine when a vulnerability was resolved since it will require all projects in the team that report the
    vulnerability to perform a scan that no longer contains the vulnerability.  
\end{itemize}




\subsection{Counting Vulnerabilities}\label{sec:counting}

\subsubsection{SAST Reported Counts}

The vulnerability counts that can be obtained via the SAST UI show counts of High, Medium, Low, and Informational vulnerabilities that have any state other
than \textbf{Not Exploitable}.  

\noindent\\As scan results are triaged, any vulnerabilities marked as \textbf{Not Exploitable} will cause the count to be recalculated
to subtract the vulnerabilities with the changed state.  This will also apply to all historical scans; viewing the summary of a historical scan
will show a count that reflects changes made to vulnerabilities that have been reported in multiple scans.  The recalculation also affects the risk score of
historical scans that is displayed in the list of scans for the project.  The recalculation is generally performed within a few minutes of the result state being recalculated.

\noindent\\The CxAnalytix Scan Summary Record currently reports a total of all vulnerabilities, regardless of the state of the vulnerability.  As of the writing of this manual,
this is considered a defect.  Future versions of CxAnalytix will calculate vulnerability counts by not considering vulnerabilities marked as \textbf{Not Exploitable}.


\subsubsection{Counting with Scan Detail Record}

It is possible to recreate the same counting logic as SAST by filtering the detail records where \texttt{NodeId == 1} and \texttt{State != 1}.  The \texttt{State}
numeric values correspond to the following result states:


\begin{itemize}
    \item 0 = To Verify
    \item 1 = Not Exploitable
    \item 2 = Confirmed
    \item 3 = Urgent
    \item 4 = Proposed Not Exploitable
\end{itemize}

\noindent\\It is often the case that there is a need to calculate triaged vs. untriaged vulnerability counts.  The triage state \texttt{To Verify} is the
initial state for all vulnerabilities upon detection.  The state value of each vulnerability can be used as filtering or grouping options to derive appropriate counts.


\subsubsection{Counting by Fingerprint}

As noted in Section \ref{sec:fingerprint}, the fingerprint can coalesce similar vulnerabilities into a single fingerprint.  Some opinions may considered this problematic for
counting purposes. 

\noindent\\The reason for coalescing is usually due to a duplicate \texttt{SimilarityId} for multiple reported data flow paths. Fingerprinted vulnerabilities coalesce into a 
single vulnerability when duplicate \texttt{SimilarityId} values cause the fingerprint calculation to yield the same fingerprint.  Vulnerabilities with the same 
\texttt{SimilarityId} are treated as the same vulnerability in SAST, even if each reported vulnerability flows through different code or exists in different files.  
This is mostly observed when triage changes for a vulnerability propagate across all projects on a team.

\noindent\\There are many reasons why duplicate \texttt{SimilairtyId} values will be generated. Copy/paste code, while not common in most code, can generate duplicate 
\texttt{SimilarityId} data flows.  The duplicate data flows often occur when:

\begin{itemize}
    \item The source and sink nodes for multiple distinct paths are the same.
    \item The source and sink nodes contain code that is the same.
\end{itemize}

\noindent\\As an example, consider this code sample that produces 3 SQL Injection vulnerabilties:

\begin{code}{}{}{}
using System;
using System.Collections.Generic;
using System.Linq;
using System.Threading.Tasks;
using Microsoft.AspNetCore.Mvc;
using Microsoft.AspNetCore.Mvc.RazorPages;
using System.Data.SqlClient;
using System.IO;
using System.Text;

namespace webapp.Pages
{
    public class IndexModel : PageModel
    {

        private String getField(String[] fields, int index)
        {
            return Request.Form[fields[index]];
        }


        public void OnPost()
        {

            String[] fieldNames = {"A", "B", "C"};
            int curField = 0;
                        
            SqlConnection con = new SqlConnection();

            String response = String.Empty;
            String dbData = null;
            
            dbData = new SQLCommand($"SELECT * FROM SomeTable WHERE SomeColumn = '{getField(fieldNames, curField)}'", con).ExecuteScalar();
            if (dbData != null)
              response += $"Field A: {dbData}<P>";
            
            curField++;

            dbData = new SQLCommand($"SELECT * FROM SomeTable WHERE SomeColumn = '{getField(fieldNames, curField)}'", con).ExecuteScalar();
            if (dbData != null)
              response += $"Field B: {dbData}<P>";

            curField++;

            dbData = new SQLCommand($"SELECT * FROM SomeTable WHERE SomeColumn = '{getField(fieldNames, curField)}'", con).ExecuteScalar();
            if (dbData != null)
              response += $"Field C: {dbData}<P>";
            
            Response.Body.Write(new ReadOnlySpan<byte>(Encoding.UTF8.GetBytes (response.ToArray ()) ) );
        }
    }
}

\end{code}

\noindent\\The SQL Injection vulnerabilties have the same source line and a unique sink line, but the \texttt{SimilarityId} is the same for all vulnerabilities.  
This means that a triage change for only one vulnerabilitiy will apply it to all vulnerabilities with the same \texttt{SimilarityId}.  Reviewing the XML report of the 
code example confirms the \texttt{SimilarityId} is the same for all SQL Injection data flow paths:

\includegraphics[scale=.6]{graphics/Data-Analysis-FAQ-SimId-XML.png}

\subsection{Counting New/Resolved/Recurrent Vulnerabilities}

To obtain only deltas of vulnerability counts between scans to track changes, uniquely identifying vulnerabilities is likely not required.  Section \ref{sec:counting}
details various methods of counting vulnerabilities.  Applying the counting logic to the latest scan and the previous scan is one method of yielding data needed to
calculate the change delta between scans.


\subsubsection{Calculating New Vulnerability Counts}

The \texttt{Status} field in the SAST Vulnerability Details record will have the value \textbf{New} to indicate the vulnerability has been detected for the first time in 
a scan for the project.

\noindent\\\texttt{Status} with the value of \textbf{Recurrent} means that this vulnerability has been reported previously for one or more scans in the same project.  
The vulnerability could be added to the most recent scan but still have the \textbf{Recurrent} status under some conditions:

\begin{itemize}
    \item The vulnerability was previously remediated and then re-introduced into the code.
    \item Preset changes removed a query from previous scans before being changed again to add the query back into the latest scans.
    \item Exclusion adjustments change the scope of scanned code and removed the vulnerable code from a prior scan before being adjusted again to add the code back
    into recent scans.
\end{itemize}


\subsection{Determining when a Vulnerability was First Detected}

The easiest method is to find the vulnerability where the \texttt{Status} field is \textbf{New}.  This works if and 
only if a sample was recorded the first time the vulnerability was detected.  There are various scenarios where this may not happen:

\begin{itemize}
    \item The report for the scan could not be retrieved at the time CxAnalytix performed the crawl for scans.
    \item Data retention has been run and the first scan was purged prior to CxAnalytix crawling the scans.
\end{itemize}

A more general method may be to use the compound identifier for tracking vulnerabilities across scans and determine which scan is associated with the 
sample containing the earliest value in the \texttt{ScanFinished} field.

\subsubsection{FirstDetectionDate}

As of SAST 9.3 and CxAnalytix 1.3.1, the field \texttt{FirstDetectionDate} is part of the data output specification.  Scans executed prior to 9.3 will not have a true
value for \texttt{FirstDetectionDate}.  


\subsection{Detecting when a Vulnerability has been Resolved}

This depends on how your organization defines the criteria for a "resolved vulnerability".  There will be two methods of determining how to find the date
for vulnerability resolution that should fit for most definitions of a "resolved vulnerability".


\noindent\\To explain, some variable definitions are required:\\
\begin{itemize}
    \item Let $V_T$ be the vulnerability that is tracked across multiple scans using the chosen composite identifier.
    \item Let $\mathbb{S}$ be the set of scans having the same \texttt{ProjectId} field value where at least one scan reports $V_T$.
    \item Let the subset $\mathbb{S}_{found}$ be the subset of scans where $V_T$ is reported
    such that $\mathbb{S}_{found} = \{\mathbb{S} | V_T \text{ is reported}\}$ and $\mathbb{S}_{found}\subseteq\mathbb{S}$
\end{itemize}

\noindent\\Finding the date $V_T$ first appeared means finding scan
$S_{found}\in\mathbb{S}_{found}$
with the earliest value for \texttt{ScanFinished}.

\subsubsection{The Easy Method}

Given the subset of scans where $V_T$ is not reported 
$\mathbb{S}_{fixed}= \{\mathbb{S} | \text{not reporting }V_T\}$ 
we know that if $\mathbb{S}_{fixed} == \varnothing$ (empty set) that the vulnerability is still outstanding.  

\noindent\\If the most recent scan $\text{S}_{latest}\in\mathbb{S}$ is also in $\mathbb{S}_{fixed}$ ($\text{S}_{latest}\in\mathbb{S}_{fixed}$), then we can find the scan
$\text{S}_{fixed}\in\mathbb{S}_{fixed}$ with the earliest \texttt{ScanFinished} date to find the date
the vulnerability was remediated.

\subsubsection{The Hard Method}

Note that it is possible for $V_T$ to be re-introduced to the code; while it may be rare, the result is that there are potentially multiple 
resolution dates. If $\text{S}_{fixed}\notin\mathbb{S}_{fixed}$, it can be assumed that the vulnerability was re-introduced and is still outstanding. 

\noindent\\The detection method presented above will technically work for all cases at the expense of the accuracy of dates related to appearance and resolution.  Your organization
can decide how they would like to approach analysis for this case.  If there is a need to find a more exact date of resolution, more advanced logic is needed.

\noindent\\For a basic method of dealing with vulnerability reappearance, the \texttt{ScanFinished} date for 
$\text{S}_{found}$ may still be considered the date $V_T$ first appeared for most tracking purposes. It must still hold 
that $\text{S}_{latest}\in\mathbb{S}_{fixed}$ to indicate the vulnerability has been resolved.

\noindent\\Using the scan $\text{S}_{most-recent-found}\in\mathbb{S}_{found}$ where the \texttt{ScanFinished} value is the most recent is the date where the search for
the latest fix date can begin.

\noindent\\Find the scan where $V_T$ was most recently fixed $\text{S}_{most-recent-fixed}\in\mathbb{S}_{fixed}$
by selecting 
$\text{S}_{most-recent-fixed}$ with a \texttt{ScanFinished} value greater than that of 
the \texttt{ScanFinished} value of $\text{S}_{most-recent-found}$
\textbf{and} the earliest value for all scans
$\text{S}\in\mathbb{S}_{fixed}$.
The \texttt{ScanFinished} value for $\text{S}_{most-recent-fixed}$ is the latest date on which
$V_T$ was resolved.

\subsubsection{Additional Considerations}

As the code changes from scan to scan, it is possible the fields used in creating a fingerprint for the vulnerability may also change.  Changes in the fingerprint may 
lead to the assumption that the vulnerability is "closed" since the vulnerability no longer appears in the scan.  A new vulnerability will then appear as "open" given 
the new fingerprint has appeared.

\noindent\\SAST assigns the \texttt{FirstDetectionDate} the \texttt{SimilarityId} of the result.  A scan may have multiple vulnerabilities reported having the 
same \texttt{SimilarityId}, therefore these results have the same \texttt{FirstDetectionDate}.  

\noindent\\Methods for tracking the lifecycle of the vulnerability for the purpose of setting a resolution SLA may need to consider this information to understand
if SLA aging resets as code changes.  In many cases, the use of the fingerprint is sufficient given that code may not change often enough to perpetually reset SLAs.


\subsection{Project Information Records}

The Project Information record is a sample of the current state of a project.  The fields indicate the state of the project at the time CxAnalytix performed the 
crawl each project.  Often the project information does not change between scans, therefore it will appear as if the project information is duplicated.

\noindent\\If a project has had no scans executed since the previous crawl, there is effectively no change that has been imposed on the project.  If there are
no changes for the project, there is no project information recorded.  If one or more
scans are executed since the previous crawl, the a Project Information sample will be recorded for the crawl.



\part{Appendices}
\appendix
\chapter{Release Notes}

\let\sectionold\thesection

\renewcommand\thesection{v}

\section{2.0.1}

\subsection*{FEATURES}
    \begin{itemize}
        \item Issue \#129 - New record type: scan statistics\\
		\indent A scan statistics record type has been added for optionally storing the scan statistics for SAST 9.4+ scans.
	
        \item Issue \#157 - include branching information for projects in CxAnalytix
		\indent Added fields in the Project Information Record:
        \begin{itemize}
			\item IsBranched
			\item BranchedAtScanId
			\item BranchParentProject
        \end{itemize}
			
	    \item Issue \#182 - Scan custom fields\\
		\indent Scan custom fields, if available, are now exported in the scan summary record.    
    \end{itemize}

\subsection*{UPDATES}
    \begin{itemize}
        \item Additional fields added to some records to aid in filtering when record types include records from multiple Checkmarx products.  
        This includes but is not limited to:
        \begin{itemize}
            \item ScanProduct
            \item ScanType
        \end{itemize}
        \item Issue \#93 - Documentation update regarding decrypting configuration sections encrypted by another user
    \end{itemize}

\subsection*{BUG FIXES}
    \begin{itemize}
        \item Issue \#158 - SAST scan summary totals should match the project state totals
        \item Issue \#183 - Audit table crawl throws an exception if the connection string is not defined
        \item Issue \#184 - Authentication issue - can't use disposed object
    \end{itemize}


\section{2.0.0}

\subsection*{FEATURES}
    \begin{itemize}
        \item Issue 12 - SCA compatibility
        \begin{itemize}
            \item The data field specification reflects data fields used by OSA and/or SCA.  The data concepts in OSA and SCA 
            are slightly different, thus the data fields will reflect these differences.
        \end{itemize}
    \end{itemize}

\subsection*{UPDATES}
    \begin{itemize}
        \item Using .Net 6.0
            \begin{itemize}
                \item Standalone execution has been supported for several versions, it is now the default.  A separate installation of .Net Core is no longer required.
            \end{itemize}
        \item Configuration file is now `cxanalytix.config` for all executables.
        \item Configuration file search path has changed to allow for easier upgrades.
        \item Configuration format for 2.x is not backwards compatible with 1.x configuration files.  Users of CxAnalytix prior to version 2.x will need 
        to create a new configuration file.
        \item Transformers are now pluggable modules that must be selected in the configuration.
        \item Outputs have always been pluggable modules, but selection is now more user friendly.
    \end{itemize}

\subsection*{BUG FIXES}
    \begin{itemize}
        \item Garbage collection tuning.
    \end{itemize}


\section{1.3.3}
\subsection*{BUG FIXES}
    \begin{itemize}
        \item Performance fix - throttle the API I/O calls during scan crawl resolution to use only the configured number of concurrent threads.
        \item Issue 142 - Correct the SinkFileName, SinkLine, SinkColumn values in the scan detail output.
        \item Issue 135 - Avoid repeatedly calling OSA endpoints if OSA is not licensed.
        \item Issue 109 - The user agent now shows API requests with CxAnalytix and version in the user agent string.
    \end{itemize}

\subsection*{UPDATES}
    \begin{itemize}
        \item As of v1.3.3, CxAnalytix is no longer compatible with SAST versions prior to 9.0.
    \end{itemize}

\section{1.3.2}
\subsection*{BUG FIXES}
    \begin{itemize}
        \item Memory leak in M\&O client REST API code fixed.
        \item Added the `RetryLoop` configuration to allow retries after timeout.
        \item Stopped the attempt to load policies at startup if the M\&O URL is not provided.
        \item Stability fixes for AMQP outputs.
        \item Dependency upgrades.
        \item Garbage collection tuning.
    \end{itemize}


\section{1.3.1}
\subsection*{FEATURES}
    \begin{itemize}
        \item Platform-specific tarballs are now created.  This is to address the dynamic loading of DPAPI that .Net Core apparently doesn't handle well in cross-platform builds.
        \item Pseudo-transactions are now off by default.
        \item New data fields added to scan summary and scan detail records.
    \end{itemize}

\subsection*{BUG FIXES}
    \begin{itemize}
        \item Issue 85 - Malformed AMQP config written on first run, preventing subsequent runs without removing the malformed config and commenting out the AMQP config class references.
    \end{itemize}


\section{1.3.0}
\subsection*{FEATURES}
    \begin{itemize}
        \item Issue 10 - Output can now be routed to AMQP endpoints
    \end{itemize}


\section{1.2.5}
\subsection*{FEATURES}
    \begin{itemize}
        \item Issue 52 - Transactional writes have been implemented as Pseudo Transactions.
    \end{itemize}

\subsection*{BUG FIXES}
    \begin{itemize}
        \item An issue with crawls aborting on SAST systems not licensed for OSA was re-introduced in 1.2.2 and has been fixed.
    \end{itemize}

\section{1.2.4}
\subsection*{BUG FIXES}
    \begin{itemize}
        \item Stability fix for cases where M\&O did not return policy violations as expected
        \item Build change to not build self-contained; this was causing issues on some Linux distros
    \end{itemize}


\section{1.2.3}
\subsection*{FEATURES}
    \begin{itemize}
        \item Issue 57 - Filtering scans crawled via Team and Project regex matching
    \end{itemize}
\subsection*{BUG FIXES}
    \begin{itemize}
        \item Issue 17 - Updated the docker image to better support persisting the state files
    \end{itemize}


\section{1.2.2}
\subsection*{FEATURES}
    \begin{itemize}
        \item Fields added to the output records
        \begin{itemize}
            \item Project Information
            \begin{itemize}
                \item LastCrawlDate
            \end{itemize}
        \end{itemize}
        \begin{itemize}
            \item Policy Violation Details
            \begin{itemize}
                \item ViolationId
            \end{itemize}
        \end{itemize}
        \item A basic [regression testing utility](https://github.com/checkmarx-ts/CxAnalytix/wiki/Development-Home) was added to test that data extraction is consistent between versions.  This is primarily targeted for developer use.
    \end{itemize}

\subsection*{BUG FIXES}
    \begin{itemize}
        \item Issue 51 - Timestamp of date to check for last scan is recorded as the finish date of the last scan found during the current crawl rather than the date of the current crawl.
        \item Issue 53 - Authorization token refresh improvements
        \item Stealth fix during development - NodeLine would be excluded from the SAST Vulnerability Details record under certain conditions 
    \end{itemize}


\section{1.2.1}
\subsection*{BUG FIXES}
    \begin{itemize}
        \item Issue 60 - A DB table row with a column containing a NULL value threw an exception and caused the DB crawl to end prematurely.
    \end{itemize}


\section{1.2.0}
\subsection*{FEATURES}
    \begin{itemize}
        \item New feature to extract audit events by crawling audit log tables in CxActivity and CxDB.  This feature is limited to use in systems that can make a connection directly to the CxSAST DB.
    \end{itemize}

\section{1.1.7}
\subsection*{BUG FIXES}
    \begin{itemize}
        \item Issue 31 - No time delay between queries for report generation status.
    \end{itemize}


\section{1.1.6}
\subsection*{BUG FIXES}
    \begin{itemize}
        \item Issue 26 - OSA scan details incomplete or missing
    \end{itemize}
\subsection*{FEATURES}
    \begin{itemize}
        \item The rolling file log naming convention should cause daily log rotation as well as 100MB max log file sizes by default.
    \end{itemize}

\section{1.1.5}
\subsection*{FEATURES}
    \begin{itemize}
        \item Added the ability to dump all network I/O to the application log.
        \item Improved error handling and exception logging for troubleshooting purposes.
    \end{itemize}
\subsection*{BUG FIXES}
    \begin{itemize}
        \item Issues 21, 22 - Networking implementation caused issues on some versions of Windows server.
    \end{itemize}


\section{1.1.4}
\subsection*{FEATURES}
    \begin{itemize}
        \item Added EngineStart/EngineFinished fields to the scan summary; no-change scans will be indicated with DateTime.MinValue
    \end{itemize}
\subsection*{BUG FIXES}
    \begin{itemize}
        \item Issue 20: Date parsing error in non-US locale
    \end{itemize}

\section{1.1.3}
\subsection*{BUG FIXES}
    \begin{itemize}
        \item Issue 18: Error when attempting to retrieve policy violation data from SAST 9.0
    \end{itemize}


\section{1.1.2}
\subsection*{FEATURES}
    \begin{itemize}
        \item Dockerfile now available as a release artifact
        \item Docker base image pushed to Docker Hub as part of the build 
    \end{itemize}

\section{1.1.1}
\subsection*{FEATURES}
    \begin{itemize}
        \item Issue 9: Resolve config values from environment variables (see the Wiki for [CxConnection](https://github.com/checkmarx-ts/CxAnalytix/wiki/CxConnection), [CxCredentials](https://github.com/checkmarx-ts/CxAnalytix/wiki/CxCredentials), and [CxAnalyticsService](https://github.com/checkmarx-ts/CxAnalytix/wiki/CxAnalyticsService))
    \end{itemize}
\subsection*{BUG FIXES}
    \begin{itemize}
        \item Issue 6: Now compatible with SAST 9.0
    \end{itemize}

\section{1.1.0}
\subsection*{FEATURES}
    \begin{itemize}
        \item Issue 4: MongoDB is now available as an output destination.
        \item Issue 5: Add instance identifier to each record.
        \item Issue 7: Add project custom fields to the output.
    \end{itemize}

\section{1.0.0 - Initial Release}
\subsection*{FEATURES}
    \begin{itemize}
        \item Output to flat log files
        \item Support for CxSAST 8.9 APIs
    \end{itemize}


\let\thesection\sectionold

\chapter{Data Field Specification}\label{chap:spec}

This chapter details the data fields output in each record type by Checkmarx product. Some fields may be omitted if they are not provided
by the source data.  (e.g. OSA records or OSA related fields would not be included in a record output if a project was not being scanned
using OSA.)  Data records can be considered a snapshot of the state of the vulnerabilities at the time the scan data was crawled.


\input{spec_sast.tex}
\subsection{Fields for Outputs Generated from Checkmarx SCA}


\subsubsection{Record: SCA Scan Summary}

\begin{itemize}

    \item HighVulnerabilityLibraries
    \item InstanceId \textit{(Only included if an instance id is configured)}
    \item LegalHigh
    \item LegalLow
    \item LegalMedium
    \item LegalUnknown
    \item LowVulnerabilityLibraries
    \item MediumVulnerabilityLibraries
    \item NonVulnerableLibraries
    \item PoliciesViolated
    \item PolicyViolations
    \item ProjectId
    \item ProjectName
    \item RulesViolated
    \item ScanFinished
    \item ScanId
    \item ScanOrigin
    \item ScanProduct
    \item ScanStart
    \item ScanType
    \item TAG\_\{tag name\} \textit{Dynamically generated if scan is tagged}
    \item TeamName
    \item TotalDirectDependencies
    \item TotalExploitablePaths
    \item TotalHighVulnerabilities
    \item TotalLegalRiskPackages
    \item TotalLibraries
    \item TotalLowVulnerabilities
    \item TotalMediumVulnerabilities
    \item VulnerabilityScore
    \item VulnerableAndOutdated
    \item VulnerableAndUpdated
\end{itemize}


\subsubsection{Record: SCA Vulnerability Details}

\begin{itemize}
    \item CVEDescription
    \item CVEName
    \item CVEPubDate
    \item CVEScore
    \item CVEUrl
    \item CVSS\_AttackComplexity
    \item CVSS\_AttackVector
    \item CVSS\_Availability
    \item CVSS\_Confidentiality
    \item CVSS\_Score
    \item CVSS\_Severity
    \item CVSS\_Version
    \item CWE
    \item ExploitableMethods
    \item InstanceId \textit{(Only included if an instance id is configured)}
    \item LibraryId
    \item LibraryLatestReleaseDate
    \item LibraryLatestVersion
    \item LibraryLegalRisk\_\{License \& Version\} \textit{(Field name is dynamically generated)}
    \item LibraryLicenses
    \item LibraryName
    \item LibraryReleaseDate
    \item LibraryVersion
    \item ProjectId
    \item ProjectName
    \item ScanFinished
    \item ScanId
    \item ScanProduct
    \item ScanRiskSeverity
    \item ScanType
    \item State
    \item TeamName
    \item Type
    \item VulnerabilityId
\end{itemize}


\subsubsection{Record: Project Information}

\begin{itemize}
    \item InstanceId \textit{(Only included if an instance id is configured)}
    \item LastCrawlDate
    \item Policies
    \item Preset
    \item ProjectId
    \item ProjectName
    \item SCA\_LastScanDate \textit{(Requires OSA)}
    \item SCA\_Scans \textit{(Requires OSA)}
    \item TAG\_\{tag name\} \textit{Dynamically generated if project is tagged}
    \item TeamName
\end{itemize}



\subsubsection{Record: Policy Violations Details}
\begin{itemize}
    \item InstanceId \textit{(Only included if an instance id is configured)}
    \item PolicyName
    \item ProjectId
    \item ProjectName
    \item RuleName
    \item RuleType
    \item ScanId
    \item ScanProduct
    \item ScanType
    \item TeamName
    \item ViolationId
    \item ViolationName
    \item ViolationOccurredDate
    \item ViolationRiskScore
    \item ViolationSeverity
    \item ViolationState
    \item ViolationStatus
\end{itemize}



\subsection{Fields for Outputs Generated from Checkmarx One}


\subsubsection{Record: SAST Scan Summary}

\begin{itemize}
    \item Branch
    \item DeepLink
    \item Engines
    \item FailedLinesOfCode
    \item FileCount
    \item High
    \item Information
    \item Initiator
    \item InstanceId \textit{(Only included if an instance id is configured)}
    \item Languages
    \item LinesOfCode
    \item Low
    \item Medium
    \item Preset
    \item ProjectId
    \item ProjectName
    \item ScanFinished
    \item ScanId
    \item ScanProduct
    \item ScanStart
    \item ScanTime
    \item ScanType
    \item SourceOrigin
    \item SourceType
    \item TAG\_\{tag name\} \textit{Dynamically generated if scan is tagged}
    \item TeamName
\end{itemize}


\subsubsection{Record: SAST Vulnerability Details}

\begin{itemize}
    \item Branch
    \item FalsePositive
    \item FirstDetectionDate
    \item InstanceId \textit{(Only included if an instance id is configured)}
    \item NodeColumn
    \item NodeFileName
    \item NodeId
    \item NodeLength
    \item NodeLine
    \item NodeName
    \item NodeType
    \item ProjectId
    \item ProjectName
    \item QueryCategories
    \item QueryCweId
    \item QueryGroup
    \item QueryId
    \item QueryLanguage
    \item QueryName
    \item QuerySeverity
    \item QueryVersionCode
    \item Remark
    \item ResultDeepLink
    \item ResultId
    \item ResultSeverity
    \item ScanFinished
    \item ScanId
    \item ScanProduct
    \item ScanType
    \item SimilarityId
    \item SinkColumn
    \item SinkFileName
    \item SinkLine
    \item State
    \item Status
    \item TeamName
    \item VulnerabilityId
\end{itemize}


\subsubsection{Record: Project Information}

\noindent\\Note that scan counts and last scan dates only reflect the scans retrieved since the last crawl.

\begin{itemize}
    \item Applications
    \item ApplicationMaxCriticality
    \item ApplicationMinCriticality
    \item CriticalityLevel
    \item SAST\_LastScanDate
    \item SAST\_Scans
    \item KICS\_LastScanDate
    \item KICS\_Scans
    \item SCA\_LastScanDate
    \item SCA\_Scans
    \item InstanceId \textit{(Only included if an instance id is configured)}
    \item LastCrawlDate
    \item Preset
    \item ProjectCreated
    \item ProjectId
    \item ProjectName
    \item ProjectUpdated
    \item RepoMainBranch
    \item RepoUrl
    \item TAG\_\{tag name\} \textit{Dynamically generated if project is tagged}
    \item TeamName
\end{itemize}


\subsubsection{Record: SCA Scan Summary}

\begin{itemize}
    \item Branch
    \item Engines
    \item HighVulnerabilityLibraries
    \item Initiator
    \item InstanceId \textit{(Only included if an instance id is configured)}
    \item LegalHigh
    \item LegalLow
    \item LegalMedium
    \item LegalUnknown
    \item LowVulnerabilityLibraries
    \item MediumVulnerabilityLibraries
    \item NonVulnerableLibraries
    \item ProjectId
    \item ProjectName
    \item ScanFinished
    \item ScanId
    \item ScanOrigin
    \item ScanProduct
    \item ScanStart
    \item ScanTime
    \item ScanType
    \item TAG\_\{tag name\} \textit{Dynamically generated if scan is tagged}
    \item TeamName
    \item TotalDirectDependencies
    \item TotalExploitablePaths
    \item TotalHighVulnerabilities
    \item TotalLegalRiskPackages
    \item TotalLibraries
    \item TotalLowVulnerabilities
    \item TotalMediumVulnerabilities
    \item VulnerabilityScore
    \item VulnerableAndOutdated
    \item VulnerableAndUpdated
\end{itemize}


\subsubsection{Record: SCA Vulnerability Details}

\begin{itemize}
    \item Branch
    \item CVEDescription
    \item CVEName
    \item CVEPubDate
    \item CVEScore
    \item CVEUrl
    \item CVSS\_AttackComplexity
    \item CVSS\_AttackVector
    \item CVSS\_Availability
    \item CVSS\_Confidentiality
    \item CVSS\_Score
    \item CVSS\_Severity
    \item CVSS\_Version
    \item CWE
    \item Engines
    \item ExploitableMethods
    \item Initiator
    \item InstanceId \textit{(Only included if an instance id is configured)}
    \item LibraryId
    \item LibraryLatestReleaseDate
    \item LibraryLatestVersion
    \item LibraryLegalRisk\_\{License \& Version\} \textit{(Field name is dynamically generated)}
    \item LibraryLicenses
    \item LibraryName
    \item LibraryReleaseDate
    \item LibraryVersion
    \item ProjectId
    \item ProjectName
    \item ScanFinished
    \item ScanStart
    \item ScanId
    \item ScanProduct
    \item ScanRiskSeverity
    \item ScanTime
    \item ScanType
    \item State
    \item TeamName
    \item Type
    \item VulnerabilityId
\end{itemize}


\subsubsection{Record: Scan Statistics}

\noindent\\In addition to the listed fields, successfully executed general query names \\
(e.g. \texttt{javaScript.Cx.General.Sanitize}) are field names in the 
document.  These are variable per scan.  The value associated with the field is the count of elements found when the general query was executed.

\noindent\\Fields prefixed with \textbf{\{language\}\_} are dynamically generated.  Only fields for languages detected by the scan are included.

\begin{itemize}
    \item \{language\}\_DomObjectCount
    \item FileFilter
    \item \{language\}\_FilesNotScanned
    \item \{language\}\_FilesParsedSuccessfullyCount
    \item \{language\}\_FilesParseFailureCount
    \item \{language\}\_FilesParsedPartiallyCount
    \item FilteredParsedLOC
    \item InstanceId \textit{(Only included if an instance id is configured)}
    \item \{language\}\_LOCParsedCount
    \item \{language\}\_LOCParseFailCount
    \item \{language\}\_LOCParseSuccessPercent
    \item PhysicalMemoryPeakMB
    \item ProjectId
    \item ProjectName
    \item ResultCount
    \item ScanId
    \item ScanProduct
    \item ScanType
    \item TeamName
    \item UnfilteredParsedLOC
    \item UnscannedFileCount
    \item VirtualMemoryPeakMB
\end{itemize}


\chapter{Shard Key Cookbook}\label{ShardKeyCookbook}


If you have come to this documentation, you may have the need to store quite a lot of data and access such data in a scalable way.  
The MongoDB documentation about \href{https://docs.mongodb.com/manual/core/sharding-shard-key/#choosing-a-shard-key}{choosing a shard key} is a good read if you are
not familiar with sharding concepts as they relate to document databases.  In the context of CxAnalytix, sharding is primarily discussed in terms of 
storage scalability.  Scalability for efficient reads (e.g. avoiding queries across multiple shards) is beyond the scope of CxAnalytix.


\section{Shard Keys and Vulnerability Data}

The difficulty in choosing a shard key from vulnerability data is that it is difficult to predict if fields have low cardinality 
(e.g. the field has very few unique values) or high cardinality (e.g. the field has mostly unique values).  Even harder still is the ability to predict
the frequency at which a field's value changes.

\noindent\\ProjectName, ProjectId, TeamName are examples of low cardinality fields within any given collection.  As scans are executed against a project, 
the names of the projects will repeat quite often.  Projects are also unlikely to change teams very often.

\noindent\\ScanId is an example of a high cardinality field.  ScanId, however, may not be an ideal selection as a shard key considering it changes
for every scan.  In an extreme scenario, imagine a data storage shard being allocated for each scan.  This may result in a system where the size of
the allocated data storage is difficult to manage.

\noindent\\Combining several fields to form a composite key can often achieve sufficient cardinality.  When choosing fields, it is important to include
fields that have a sufficient change frequency.  The required frequency may depend on the volume of scanning performed.  If the frequency of change is 
very low and scan volume very high, the shard storage space may reach maximum capacity.


\section{Generated Shard/Partition Keys}
Section \ref{sec:mongo_config} details the MongoDB configuration.  The MongoDB configuration gives the ability to optionally specify a calculated
shard key to add to each record written to a collection.  This is primarily for use with cloud-based document storage systems that dynamically
expand the storage based on a value at the root of the document.  It can be used as a generic method for automating the calculation of shard
affinity if desired, but MongoDB's shard indexing capability is more flexible than using this option.

\noindent\\Each collection of documents has different fields and cardinality considerations.  Chapter \ref{chap:spec} details the fields
available for documents written in each collection.  The fields of each data record can be used to compose the shard key via
the Key Format Specification described in Appendix \ref{chap:key_format_spec}.

\noindent\\For those that wish to experiment with the Key Format Specification, the \texttt{StringFormatExamples} project can be used
for experimentation purposes.


\section{Example Shard Key Format Specifiers}

\textbf{This section is intended to be where new examples of shard keys are documented as they are chosen in field implementations.}

\noindent\\The examples below are given for consideration of a suitable shard key.  The volume of scans in an organization should be taken into
consideration when selecting a shard key.  The given examples are likely suitable for a moderate scan volume.

\noindent\\The fields \texttt{TeamName} and \texttt{ProjectName} are common to all records and are often easy to add (either both or one)
to increase cardinality.  Date fields are also generally a good choice for increasing cardinality; using the field format specifier, the cardinality increases as the
time span length decreases (e.g. year > month > day-of-week > day-of-month and so on).


\subsection{SAST Scan Summary Example}

This example uses the scan type, the year and full name of the day of the week when the scan finished.

\begin{code}{Shard Key Example for SAST\_Scan\_Summary}{}{}
<add KeyName="pkey" CollectionName="SAST_Scan_Summary"
    FormatSpec="{ScanType}-{ScanFinished:yyyy-dddd}" NoHash="true" />
\end{code}

\subsection{SAST Vulnerability Details}

This example uses the scan type, the query group, the year and full name of the day of the week when the scan finished.

\begin{code}{Shard Key Example for SAST\_Scan\_Detail}{}{}
<add KeyName="pkey" CollectionName="SAST_Scan_Detail"
    FormatSpec="{ScanType}-{QueryGroup}-{ScanFinished:yyyy-dddd}" NoHash="true" />
\end{code}

\chapter{Key Format Specification}\label{chap:key_format_spec}

The Key Format Specification is a string composed of alphanumeric text, field specifiers, and field format specifiers.  It is used in various configuration
elements to dynamically generate a value containing record data that is added to data sent via the configured output method.

\noindent\\The syntax of a Key Format Specification is:

\noindent\\\texttt{\{field key[:format value]\}}

\noindent\\Where the \texttt{field key} element is the name of the field in the record from which to extract a value used when composing a string value.
The following example shows how to create a shard key from the data found in \texttt{ScanType}, \texttt{TeamName}, and \texttt{ProjectName}.

\noindent\\
\begin{xml}{Spec}{example with field expansion}{}
<Spec KeyName="pkey" CollectionName="SAST_Scan_Summary" 
    FormatSpec="SHARD-{ScanType}-{TeamName}-{ProjectName}"  />
\end{xml}


\noindent\\When the \texttt{field key} element is a dictionary type, a dotted notation may be used to reference a key in the dictionary
value referenced by \texttt{field key}.  The following example creates a shard key using the value of a custom field:

\begin{xml}{Spec}{example with field dotted notation}{}
<Spec KeyName="pkey" CollectionName="SAST_Scan_Summary" 
    FormatSpec="SHARD-{ScanType}-{TeamName}-{ProjectName}-{CustomFields.MyCustomField}"
    />
\end{xml}


\noindent\\The curly braces (\texttt{\{} and \texttt{\}}) can be embedded in the generated string by using a backslash (\texttt{\textbackslash}) to escape 
the curly brace.  The following example shows a shard key with contents surrounded by curly braces:

\begin{xml}{Spec}{example with escaped braces}{}
<Spec KeyName="pkey" CollectionName="SAST_Scan_Summary"
    FormatSpec="\{{ScanType}-{TeamName}-{ProjectName}\}"  />
\end{xml}

\section{Field Format Specifier}

The field format specifier follows the same convention of a .Net string format specifier.  The format string used depend on the data type:

\begin{itemize}
    \item\href{https://docs.microsoft.com/en-us/dotnet/standard/base-types/standard-date-and-time-format-strings}{Standard} and \href{https://docs.microsoft.com/en-us/dotnet/standard/base-types/custom-date-and-time-format-strings}{Custom} date and time format strings.
    \item\href{https://docs.microsoft.com/en-us/dotnet/standard/base-types/standard-timespan-format-strings}{Standard} and \href{https://docs.microsoft.com/en-us/dotnet/standard/base-types/custom-timespan-format-strings}{Custom} time span format strings.
    \item\href{https://docs.microsoft.com/en-us/dotnet/standard/base-types/standard-numeric-format-strings}{Standard} and \href{https://docs.microsoft.com/en-us/dotnet/standard/base-types/custom-numeric-format-strings}{Custom} numeric format strings.
\end{itemize}





\chapter{Forwarding Data to Splunk}\label{chap:splunk_config}

The Log4Net output is used to generate log files that are tailed and forwarded to Splunk via the Splunk Universal Forwarder.  This requires the Log4Net
output to be configured so that the Universal Forwarder can find the generated output files.  Please refer to Section \ref{sec:runtime_config} for details
about choosing the Log4Net output module.

\noindent\\The Log4Net configuration file should be modified only to change the output path of the generated data output files.  The generated data output files are different
than the application logging output files in that the data output files contain data to be used for analysis purposes. It may be desirable to also forward
the application log files to Splunk for monitoring and troubleshooting purposes.


\section{Configuring CxAnalytix for Splunk}\label{sec:splunk}

\noindent\\The \texttt{CxAnalytixService} configuration section in \texttt{cxanalytix.config} contains record name mapping attributes.  The XML example 
\hyperref[lst:record_map]{Record Map Configuration}
shows an example configuration with the record names mapped to file logger names shown configured in the example 
\hyperref[lst:record_loggers]{Log4Net Logger Configurations}.  The loggers
reference file appenders, as seen in the snippet \hyperref[lst:record_appenders]{Log4Net Record File Appenders}.  The appender configuration, 
by default, places all output in the \texttt{logs}
directory, which resolves to the current working directory set when a CxAnalytix process executes.  The location of the output files can be changed
by modifying the appender configuration in \texttt{cxanalytix.log4net}.



\begin{code}{Example Record Map Configuration}{\label{lst:record_map}}{}
<CxAnalytixService 
    ConcurrentThreads="2" 
    StateDataStoragePath="%CHECKMARX_STATE_PATH%"
    ProcessPeriodMinutes="120"
    OutputModuleName="log4net"
    SASTScanSummaryRecordName="RECORD_SAST_Scan_Summary"
    SASTScanDetailRecordName="RECORD_SAST_Scan_Detail"
    SCAScanSummaryRecordName="RECORD_SCA_Scan_Summary"
    SCAScanDetailRecordName="RECORD_SCA_Scan_Detail"
    ProjectInfoRecordName="RECORD_Project_Info"
    PolicyViolationsRecordName="RECORD_Policy_Violations">
    <EnabledTransformers>
        <Transformer Name="SAST" />
    </EnabledTransformers>
</CxAnalytixService>
\end{code}


\begin{code}{Log4Net Logger Configurations}{\label{lst:record_loggers}}{}
.. snip ..
<logger name="RECORD_SAST_Scan_Summary" aditivity="false">
    <level value="ALL" />
    <appender-ref ref="SAST_SS" />
</logger>
  .. snip ..
\end{code}


\begin{code}{Log4Net Record File Appenders}{\label{lst:record_appenders}}{}
.. snip ..
<appender name="SAST_SS" type="log4net.Appender.RollingFileAppender">
    <appendToFile value="true" />
    <maximumFileSize value="100MB" />
    <rollingStyle value="Composite" />
    <staticLogFileName value="false" />
    <countDirection value="1" />
    <file type="log4net.Util.PatternString" value="logs/sast_scan_summary" />
    <datePattern value="'.'yyyy_MM_dd'.log'" />
    <preserveLogFileNameExtension value="true" />

    <layout type="log4net.Layout.PatternLayout">
        <conversionPattern value="%message%newline" />
    </layout>
</appender>
.. snip ..
\end{code}

\section{Splunk Universal Forwarder Configuration}
The \href{https://www.splunk.com/en_us/download/universal-forwarder.html}{Splunk Universal Forwarder} is used to send data to Splunk Enterprise or Splunk Cloud.  
Please refer to the Splunk website for information for details about installing and configuring the Universal Forwarder.


\subsection{Output File Tailing Configuration}

Assuming an installed forwarder is able to connect to the desired Splunk instance, 
create the \href{https://docs.splunk.com/Documentation/Splunk/latest/Admin/Inputsconf}{\texttt{inputs.conf}} file at the appropriate location 
(e.g. \texttt{/etc/apps/splunkclouduf/default/inputs.conf}).  The example \hyperref[lst:inputsconf]{inputs.conf configuration} below shows an example of 
an \texttt{inputs.conf} file with monitoring stanzas appropriate for each type of record. 


\begin{code}{inputs.conf Configuration}{\label{lst:inputsconf}}{}
[monitor://{path to logs}\CxAnalytixService...]
sourcetype=service

[monitor://{path to logs}\sast_scan_summary...]
sourcetype=sast_scan_summary

[monitor://{path to logs}\sast_scan_detail...]
sourcetype=sast_scan_detail

[monitor://{path to logs}\sast_project_info...]
sourcetype=sast_project_info

[monitor://{path to logs}\sast_policy_violations...]
sourcetype=sast_policy_violation

[monitor://{path to logs}\sca_scan_summary...]
sourcetype=sca_scan_summary

[monitor://{path to logs}\sca_scan_detail...]
sourcetype=sca_scan_detail

[monitor://{path to logs}\CxActivity_dbo_AuditTrail...]
sourcetype=cxactivity_audittrail

[monitor://{path to logs}\CxActivity_dbo_Audit_Scans...]
sourcetype=cxactivity_auditscans

[monitor://{path to logs}\CxActivity_dbo_Audit_Reports...]
sourcetype=cxactivity_auditreports

[monitor://{path to logs}\CxActivity_dbo_Audit_Queries...]
sourcetype=cxactivity_auditqueries

[monitor://{path to logs}\CxActivity_dbo_Audit_Projects...]
sourcetype=cxactivity_auditprojects

[monitor://{path to logs}\CxActivity_dbo_Audit_Presets...]
sourcetype=cxactivity_auditpresets

[monitor://{path to logs}\CxActivity_dbo_Audit_DataRetention...]
sourcetype=cxactivity_auditdataretention
\end{code}

\subsection{Configuring the Source Types}

The source types on the Splunk server need to be configured to appropriately parse JSON.  
This can be done using \href{https://docs.splunk.com/Documentation/Splunk/latest/Admin/Propsconf}{\texttt{props.conf}} (only available in Splunk Enterprise) 
or through the Splunk UI. A source type should be created the matches each record output source types as defined in 
\href{https://docs.splunk.com/Documentation/Splunk/latest/Admin/Inputsconf}{\texttt{inputs.conf}}.  The \hyperref[lst:sourcetypes]{configuration snippet below} shows an example
of a source type entry.  The source type configuration needs to be performed for each source type.


\begin{code}{Log4Net Record File Appenders}{\label{lst:sourcetypes}}{}
LINE_BREAKER=([\r\n]+)
KV_MODE=json
TRUNCATE=0
SHOULD_LINEMERGE=false
\end{code}

\noindent\\\textbf{Extracting Timestamps}\\

\noindent\\The source data contains timestamp fields that can be used as the timestamp Splunk uses when indexing the data.  Without specifying how to extract the
timestamp from each source type, the timestamp will default to the timestamp when the data was indexed.  This may work for current data, but data searches 
will also return historical data that is outside of the selected search time frame.

\noindent\\For the SAST Scan Summary, SAST Scan Detail, SCA Scan Summary, and SCA Scan Detail source types, this configuration option should be added:

\noindent\\\texttt{TIME\_PREFIX=\^.*ScanFinished".+?"}

\noindent\\For the SAST Project Info source type, this configuration option should be added:

\noindent\\\texttt{TIME\_PREFIX=\^.*LastCrawlDate".+?"}

\noindent\\For the SAST Policy Violation source type, this configuration option should be added:

\noindent\\\texttt{TIME\_PREFIX=\^.*ViolationOccurredDate".+?"}

\noindent\\For the Audit Trail source type, this configuration option should be added:

\noindent\\\texttt{TIME\_PREFIX=\^.*EndTime".+?"}

\noindent\\For the Audit\_Scans, Audit\_Reports, Audit\_Queries, Audit\_Projects, Audit\_Presets, Audit\_DataRetention, this configuration option should be added:

\noindent\\\texttt{TIME\_PREFIX=\^.*TimeStamp".+?"}


\chapter{Troubleshooting}

\section{Configuration file missing}
If you get this error:

\begin{lstlisting}
Unhandled Exception: System.TypeInitializationException: The type initializer for 
'CxAnalytix.Configuration.Config' threw an exception. 
    ---> System.IO.FileNotFoundException: Configuration file missing.
   at CxAnalytix.Configuration.Config..cctor() in 
   c:\programdata\checkmarx\CxAnalytix\Configuration\Config.cs:line 26
   --- End of inner exception stack trace ---
   at CxAnalytixCLI.Program.Main(String[] args) in 
   c:\programdata\checkmarx\CxAnalytix\CxAnalytixCLI\Program.cs:line 47
\end{lstlisting}

\noindent\\Try setting your current directory to the same directory as the CxAnalytixCLI executable e.g.

\noindent\\\texttt{cd C:\textbackslash ProgramData\textbackslash checkmarx\textbackslash CxAnalytix\textbackslash artifacts\textbackslash Release\textbackslash}


\section{Trace Web API I/O Data}

All data I/O with the web APIs can be captured by adding a logger to the log4net configuration with the following lines:

\begin{lstlisting}
<logger name="CxRestClient.IO">
    <level value="TRACE_NETWORK" />
</logger>
\end{lstlisting}

\section{Trace Web API Operations}

\noindent\\Capturing the web API data I/O generates a large amount of logging data.  A reduced set of data showing web API requests, timings, and 
response statuses can be captured by adding the following logger to the log4net configuration:

\begin{lstlisting}
<logger name="CxRestClient.Utility">
  <level value="TRACE" />
</logger>
\end{lstlisting}

\section{Execution Logging Verbosity}

\noindent\\The \texttt{TRACE\_NETWORK} log level setting as described above is generally only useful in the 
\texttt{CxRestClient.IO} namespace.  It can also be applied at the root level logger to increase the
verbosity of logging to the entire application to the maximum level possible:

\begin{lstlisting}
<root>
    <level value="TRACE_NETWORK" />
    <appender-ref ref="Console" />
    <appender-ref ref="RollingFile" />
</root>
\end{lstlisting}

\noindent\\Applying \texttt{TRACE\_NETWORK} at the root level is really not recommended.  A level of logging that increases the verbosity of execution
logging but excludes network traffic is \texttt{TRACE}:

\begin{lstlisting}
<root>
    <level value="TRACE" />
    <appender-ref ref="Console" />
    <appender-ref ref="RollingFile" />
</root>
\end{lstlisting}

\noindent\\The \texttt{DEBUG} level of logging is less verbose than \texttt{TRACE} and should be the first choice for troubleshooting issues:

\begin{lstlisting}
<root>
    <level value="DEBUG" />
    <appender-ref ref="Console" />
    <appender-ref ref="RollingFile" />
</root>
\end{lstlisting}



\end{document}