\subsection{AMQP Output Configuration}\label{sec:amqp_config}

AMQP is a standard wire-protocol for message queueing that can be utilized by CxAnalytix to support complex architectures for utilizing crawled 
vulnerability data.  While CxAnalytix does not feed data in real-time, sending record entries as messages to an AMQP endpoint allows for some
advanced near-realtime and stream analytics applications.  AMQP supports many concepts that make it ideal to support message-bus and other
distributed data consumption architectures.


% Key Features of AMQP Output
% * Headers and routing keys can be set with static values and/or values composed of [fields](SPEC) from each record entry using a [key format specifier](ShardKeyCookbook#key-format-specifier).
% * Filtering of records can be done prior to sending records via the [Team and Project filtering](Configure-ProjectFilterRegex) configuration or via topic and/or header Exchange/Queue bindings.
% * Record fields can be filtered with Pass or Reject filter specifications for any record type.
% * Record entries for all record types can be sent to a default exchange or an exchange specifically set for the record type.
% * The Exchange and Queue topology is not dictated by CxAnalytix; complex routing of data can be configured via Exchange bindings on the broker.
%-----%


\subsubsection{AMQP Connection Configuration}

The \texttt{AMQPConnection} configuration element is used to configure connection information for AMQP endpoints.  Endpoints for all nodes in a message queue cluster
can be defined in the configuration, allowing for failover to a different node in the cluster if one node fails.  The \texttt{ClusterNodes} child element
contains one or more \texttt{Endpoint} child elements.  The \texttt{Endpoint} element can contain a single \texttt{SSLOptions} element that allows
for configuration of SSL certificate options.  The full example XML is displayed below:\\


\begin{xml}{AMQPConnection}{\expandsenv\encrypts}{}
<AMQPConnection UserName="foo" Password="bar">
    <ClusterNodes>
        <Endpoint 
            AmqpUri="amqp://localhost:5672" 
            />
        ...
        <Endpoint 
            AmqpUri="amqps://hostname:5671">
            <SSLOptions 
                AllowCertNameMismatch="false" 
                AllowCertificateChainErrors="false" 
                ServerName="remote-name" 
                ClientCertPath ="" 
                ClientCertPassphrase=""
                />
        </Endpoint>
    </ClusterNodes>
</AMQPConnection>
\end{xml}


\begin{table}[h]
    \caption{AMQPConnection Attributes}        
    \begin{tabularx}{\textwidth}{cccl}
        \toprule
        \textbf{Attribute} & \textbf{Default} & \textbf{Required} & \textbf{Description}\\
        \midrule
        \texttt{Username} & N/A & Yes & \makecell[l]{The user name used to authenticate with the AMQP endpoint.}\\
        \midrule
        \texttt{Password} & N/A & Yes & \makecell[l]{The password used to authenticate with the AMQP endpoint.}\\
        \bottomrule
    \end{tabularx}
\end{table}


\noindent\\A basic endpoint configuration is shown in the following XML.  In this example, only the AMQP endpoint URI is required.\\

\begin{xml}{AMQPConnection> \ \faArrowLeft \ <Endpoint}{\expandsenv\encrypts}{ \ (Basic)}
<Endpoint AmqpUri="amqp://localhost:5672" />
\end{xml}


\noindent\\An advanced endpoint configuration is shown in the following XML.  In this example, the AMQP endpoint URI is
configured to encrypt the data exchanged between CxAnalytix and the AMQP endpoint.  Additional SSL options are added to define how
the SSL connection is established.\\

\begin{xml}{AMQPConnection> \ \faArrowLeft \ <Endpoint}{\expandsenv\encrypts}{ \ (Advanced)}
<Endpoint
    AmqpUri="amqps://hostname:5671">
    <SSLOptions 
        AllowCertNameMismatch="false" 
        AllowCertificateChainErrors="false" 
        ServerName="remote-name" 
        ClientCertPath ="" 
        ClientCertPassphrase=""
        />
</Endpoint>
\end{xml}
    

\noindent\\The \texttt{SSLOptions} element is shown in more detail below.

\begin{xml}{AMQPConnection> \ \faArrowLeft \ <Endpoint> \ \faArrowLeft \ <SSLOptions}{\expandsenv\encrypts}{}
<SSLOptions 
    AllowCertNameMismatch="false" 
    AllowCertificateChainErrors="false" 
    ServerName="remote-name" 
    ClientCertPath ="" 
    ClientCertPassphrase=""
    />
\end{xml}
    
\begin{table}[h]
    \caption{SSLOptions Attributes}        
    \begin{tabularx}{\textwidth}{cccl}
        \toprule
        \textbf{Attribute} & \textbf{Default} & \textbf{Required} & \textbf{Description}\\
        \midrule
        \texttt{AllowCertNameMismatch} & false & No & \makecell[l]{If true, failures to match the host name\\
        in the AmqpUri to the server\\
        name in the certificate are ignored.}\\
        \midrule
        \texttt{AllowCertificateChainErrors} & false & No & \makecell[l]{If true, missing certificates in\\
        the CA chain are accepted.}\\
        \midrule
        \texttt{ServerName} & N/A & No & \makecell[l]{The name expected on\\
        the remote SSL certificate.}\\
        \midrule
        \texttt{ClientCertPath} & N/A & No & \makecell[l]{A filepath to a PEM encoded\\
        client certificate.}\\
        \midrule
        \texttt{ClientCertPassphrase} & N/A & No & \makecell[l]{The password for the PEM encoded\\
        client certificate.}\\
        \bottomrule
    \end{tabularx}
\end{table}


\subsubsection{AMQP Data Exchange Configuration}

\ref{chap:key_format_spec}

\begin{xml}{AMQPConfig}{\expandsenv}{}
<AMQPConfig DefaultExchange="bar">
    <RecordSpecs>
        <Record Name="SAST_Scan_Summary">
            <MessageHeaders>
                <Header Key="RecordType" Spec="SAST_Scan_Summary" />
                <Header Key="ScanType" Spec="{ScanType}" />
            </MessageHeaders> 
        </Record>

        <Record 
            Name="SAST_Scan_Detail" 
            Exchange="foo" 
            TopicSpec="{ScanType}.{NodeId}.{Status}.{ResultSeverity}.{QueryLanguage}.{QueryName}">
            <Filter Mode="Reject">
                <Fields>
                    <Field Name="NodeCodeSnippet"/>
                </Fields>
            </Filter>
            <MessageHeaders>
                <Header Key="RecordType" Spec="SAST_Scan_Detail" />
                <Header Key="ScanType" Spec="{ScanType}" />
            </MessageHeaders> 
        </Record>
    </RecordSpecs>
</AMQPConfig>
\end{xml}
    
