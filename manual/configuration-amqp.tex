\subsection{AMQP Output Configuration}\label{sec:amqp_config}

AMQP is a standard wire-protocol for message queueing that can be utilized by CxAnalytix to support complex architectures for utilizing crawled 
vulnerability data.  While CxAnalytix does not feed data in real-time, sending record entries as messages to an AMQP endpoint allows for some
advanced near-realtime and stream analytics applications.  AMQP supports many concepts that make it ideal to support message-bus and other
distributed data consumption architectures.




\subsubsection{AMQP Connection Configuration}

The \texttt{AMQPConnection} configuration element is used to configure connection information for AMQP endpoints.  Endpoints for all nodes in a message queue cluster
can be defined in the configuration, allowing for failover to a different node in the cluster if one node fails.  The \texttt{ClusterNodes} child element
contains one or more \texttt{Endpoint} child elements.  The \texttt{Endpoint} element can contain a single \texttt{SSLOptions} element that allows
for configuration of SSL certificate options.  The full example XML is displayed below:\\


\begin{xml}{AMQPConnection}{\expandsenv\encrypts}{}
<AMQPConnection UserName="foo" Password="bar">
    <ClusterNodes>
        <Endpoint 
            AmqpUri="amqp://localhost:5672" 
            />
        ...
        <Endpoint 
            AmqpUri="amqps://hostname:5671">
            <SSLOptions 
                AllowCertNameMismatch="false" 
                AllowCertificateChainErrors="false" 
                ServerName="remote-name" 
                ClientCertPath ="" 
                ClientCertPassphrase=""
                />
        </Endpoint>
    </ClusterNodes>
</AMQPConnection>
\end{xml}

\noindent\\Note that a single endpoint definition to connect to an AMQP cluster behind a load balancer may be sufficient for most CxAnalytix installations.

\begin{table}[h]
    \caption{AMQPConnection Attributes}        
    \begin{tabularx}{\textwidth}{cccl}
        \toprule
        \textbf{Attribute} & \textbf{Default} & \textbf{Required} & \textbf{Description}\\
        \midrule
        \texttt{Username} & N/A & Yes & \makecell[l]{The user name used to authenticate with the AMQP endpoint.}\\
        \midrule
        \texttt{Password} & N/A & Yes & \makecell[l]{The password used to authenticate with the AMQP endpoint.}\\
        \bottomrule
    \end{tabularx}
\end{table}


\noindent\\A basic endpoint configuration is shown in the following XML.  In this example, only the AMQP endpoint URI is required.\\

\begin{xml}{AMQPConnection> \ \faArrowLeft \ <Endpoint}{\expandsenv\encrypts}{ \ (Basic)}
<Endpoint AmqpUri="amqp://localhost:5672" />
\end{xml}


\noindent\\An advanced endpoint configuration is shown in the following XML.  In this example, the AMQP endpoint URI is
configured to encrypt the data exchanged between CxAnalytix and the AMQP endpoint.  Additional SSL options are added to define how
the SSL connection is established.\\

\begin{xml}{AMQPConnection> \ \faArrowLeft \ <Endpoint}{\expandsenv\encrypts}{ \ (Advanced)}
<Endpoint
    AmqpUri="amqps://hostname:5671">
    <SSLOptions 
        AllowCertNameMismatch="false" 
        AllowCertificateChainErrors="false" 
        ServerName="remote-name" 
        ClientCertPath ="" 
        ClientCertPassphrase=""
        />
</Endpoint>
\end{xml}
    

\noindent\\The \texttt{SSLOptions} element is shown in more detail below.

\begin{xml}{AMQPConnection> \ \faArrowLeft \ <Endpoint> \ \faArrowLeft \ <SSLOptions}{\expandsenv\encrypts}{}
<SSLOptions 
    AllowCertNameMismatch="false" 
    AllowCertificateChainErrors="false" 
    ServerName="remote-name" 
    ClientCertPath ="" 
    ClientCertPassphrase=""
    />
\end{xml}
    
\begin{table}[h]
    \caption{SSLOptions Attributes}        
    \begin{tabularx}{\textwidth}{cccl}
        \toprule
        \textbf{Attribute} & \textbf{Default} & \textbf{Required} & \textbf{Description}\\
        \midrule
        \texttt{AllowCertNameMismatch} & false & No & \makecell[l]{If true, failures to match the host name\\
        in the AmqpUri to the server\\
        name in the certificate are ignored.}\\
        \midrule
        \texttt{AllowCertificateChainErrors} & false & No & \makecell[l]{If true, missing certificates in\\
        the CA chain are accepted.}\\
        \midrule
        \texttt{ServerName} & N/A & No & \makecell[l]{The name expected on\\
        the remote SSL certificate.}\\
        \midrule
        \texttt{ClientCertPath} & N/A & No & \makecell[l]{A filepath to a PEM encoded\\
        client certificate.}\\
        \midrule
        \texttt{ClientCertPassphrase} & N/A & No & \makecell[l]{The password for the PEM encoded\\
        client certificate.}\\
        \bottomrule
    \end{tabularx}
\end{table}


\subsubsection{AMQP Data Exchange Configuration}

The \texttt{AMQPConfig} allows for some advanced routing configurations when forwarding data to an AMQP endpoint.  Crawled vulnerability
data is sent to an AMQP Exchange that will allow complex Exchange/Queue topologies to route the data to the appropriate consumer.  Some configuration
elements can form values injected into data sent to the AMQP endpoint using the Key Format Specification as described in Appendix \ref{chap:key_format_spec}.

\noindent\\The example XML below shows a complex example to illustrate some of the capabilities of adding data to messages sent to the AMQP endpoint.  
An AMQP configuration can add headers and routing keys with static or dynamic values.  These can be used with different types of AMQP Exchanges
to control how messages are eventually routed to the consumer of the messages.  

\par{\noindent\\The \texttt{AMQPConfig} Element}

\noindent\\Starting with the \texttt{AMQPConfig} element, the \texttt{DefaultExchange} attribute is an optional attribute.  If there is a value assigned to this attribute,
it is the name of the AQMP Exchange where all messages will be placed unless the name of an Exchange is set in a \texttt{Record} sub-element.



\begin{xml}{AMQPConfig}{\expandsenv\contentvariables}{}
<AMQPConfig DefaultExchange="bar">
    <RecordSpecs>
        <Record Name="SAST_Scan_Summary">
            <MessageHeaders>
                <Header Key="RecordType" Spec="SAST_Scan_Summary" />
                <Header Key="ScanType" Spec="{ScanType}" />
            </MessageHeaders> 
        </Record>

        <Record 
            Name="SAST_Scan_Detail" 
            Exchange="foo" 
            TopicSpec="{ScanType}.{NodeId}.{Status}.{ResultSeverity}.{QueryLanguage}.{QueryName}">
            <Filter Mode="Reject">
                <Fields>
                    <Field Name="NodeCodeSnippet"/>
                </Fields>
            </Filter>
            <MessageHeaders>
                <Header Key="RecordType" Spec="SAST_Scan_Detail" />
                <Header Key="ScanType" Spec="{ScanType}" />
            </MessageHeaders> 
        </Record>
    </RecordSpecs>
</AMQPConfig>
\end{xml}
    


\par{\noindent\\The \texttt{Record} Element}

\noindent\\One or more \texttt{Record} elements can be created as children of the \texttt{RecordSpecs} element.  The example XML below shows a \texttt{Record}
element configured to sent records to a specific AMQP Exchange, using a dynamically generated topic, and omitting some of the fields in the data record.


\begin{xml}{Record}{\expandsenv\contentvariables}{}
<Record 
    Name="SAST_Scan_Detail" 
    Exchange="foo" 
    TopicSpec="{ScanType}.{NodeId}.{Status}.{ResultSeverity}.{QueryLanguage}.{QueryName}">
    <Filter Mode="Reject">
        <Fields>
            <Field Name="NodeCodeSnippet"/>
        </Fields>
    </Filter>
    <MessageHeaders>
        <Header Key="RecordType" Spec="SAST_Scan_Detail" />
        <Header Key="ScanType" Spec="{ScanType}" />
    </MessageHeaders> 
</Record>
\end{xml}



\begin{table}[h]
    \caption{Record Attributes}        
    \begin{tabularx}{\textwidth}{cccl}
        \toprule
        \textbf{Attribute} & \textbf{Default} & \textbf{Required} & \textbf{Description}\\
        \midrule
        \texttt{Name} & N/A & Yes & \makecell[l]{The name of the record (configuration described\\
        in Section \ref{sec:runtime_config}).\\
        }\\
        \midrule
        \texttt{Exchange} & \makecell[c]{See\\Description} & No & \makecell[l]{The name of the exchange to use for this record type.\\
        If this attribute is not provided, the exchange named\\
        as the \texttt{DefaultExchange} is used.\\
        }\\
        \midrule
        \texttt{TopicSpec} & N/A & No & \makecell[l]{A specification for forming a topic by combining\\
        static values and/or values from the record\\
        itself. This is commonly used to enable\\
        routing of messages using an AMQP Topic Exchange.\\
        }\\
        \bottomrule
    \end{tabularx}
\end{table}



\par{\noindent\\The \texttt{Filter} Element}


\noindent\\The optional \texttt{Filter} element can be used to change the data fields that are sent to the AMQP endpoint
as messages specified in the parent \texttt{Record} element. There are two modes:


\begin{itemize}
    \item \texttt{Reject} - In this mode, \textbf{all} named fields are omitted from the message sent for the record type.
    \item \texttt{Pass} - In this mode, \textbf{only} named fields are included in the message sent for the record type.
\end{itemize}

\noindent\\If this element is not included, all fields for the parent record type are sent as messages to the AMQP endpoint.  The XML example
below shows that the \texttt{NodeCodeSnippet} field is omitted from the message transmitted to the AMQP endpoint by using the \texttt{Reject}
filter mode.

\begin{xml}{Filter}{\expandsenv\contentvariables}{}
<Filter Mode="Reject">
    <Fields>
        <Field Name="NodeCodeSnippet"/>
    </Fields>
</Filter>
\end{xml}
    


\par{\noindent\\The \texttt{MessageHeaders} Element}

\noindent\\The \texttt{MessageHeaders} element allows for custom headers to be added to each message before it is transmitted
to the AMQP endpoint.  This is often used for AMQP Exchanges that route messages using header values.  In the XML example below, 
two headers are added to each message:


\begin{itemize}
    \item Header \texttt{RecordType} with the static value "SAST\_Scan\_Detail".  
    \item Header \texttt{ScanType} with the dynamic value added from the record's \texttt{ScanType} field.  
\end{itemize}


\begin{xml}{MessageHeaders}{\expandsenv\contentvariables}{}
<MessageHeaders>
    <Header Key="RecordType" Spec="SAST_Scan_Detail" />
    <Header Key="ScanType" Spec="{ScanType}" />
</MessageHeaders> 
\end{xml}
