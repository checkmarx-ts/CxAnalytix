\section{General Configuration}\label{sec:general}

\subsection{Never Do This}

The \texttt{cxanalytix.config} file needs to be modified to supply the appropriate configuration elements.  The \texttt{configSections}
element is not intended to be user configurable.  The XML snippet \hyperref[lst:motouch]{"Part of the Configuration File to Never Change"} shows an example of the contents of this XML element;
\textbf{never change anything in this section.}

\begin{code}{Part of the Configuration File to Never Change}{\label{lst:motouch}}{}
<configSections>
    <section name="CxAnalytixService" 
        type="CxAnalytix.Configuration.Impls.CxAnalytixService, Configuration" />
    <section name="CxSASTCredentials" 
        type="CxAnalytix.Configuration.Impls.CxCredentials, Configuration" />
    <section name="CxSCACredentials" 
        type="CxAnalytix.Configuration.Impls.CxMultiTenantCredentials, Configuration" />
    <section name="CxSASTConnection" 
        type="CxAnalytix.Configuration.Impls.CxSASTConnection, Configuration" />
    <section name="CxSCAConnection" 
        type="CxAnalytix.XForm.ScaTransformer.Config.CxScaConnection, ScaTransformer" />
    <section name="ProjectFilterRegex" 
        type="CxAnalytix.Configuration.Impls.CxFilter, Configuration"/>
    <section name="CxAuditTrailSuppressions" 
        type="CxAnalytix.AuditTrails.Crawler.Config.CxAuditTrailSuppressions, CxAuditTrailsCrawler"/>
    <section name="CxAuditTrailRecords" 
        type="CxAnalytix.AuditTrails.Crawler.Config.CxAuditTrailRecordNameMap, CxAuditTrailsCrawler"/>
    <section name="CxDB" 
        type="CxAnalytix.CxAuditTrails.DB.Config.CxAuditDBConnection, CxAuditTrailsDB"/>
    <section name="AMQPConnection" 
        type="CxAnalytix.Out.AMQPOutput.Config.Impls.AmqpConnectionConfig, AMQPOutput"/>
    <section name="AMQPConfig" 
        type="CxAnalytix.Out.AMQPOutput.Config.Impls.AmqpConfig, AMQPOutput"/>
    <section name="CxLogOutput" 
        type="CxAnalytix.Out.Log4NetOutput.Config.Impl.LogOutputConfig, Log4NetOutput" />
    <section name="CxMongoOutput" 
        type="CxAnalytix.Out.MongoDBOutput.Config.Impl.MongoOutConfig, MongoDBOutput" />
    <section name="CxMongoConnection" 
        type="CxAnalytix.Out.MongoDBOutput.Config.Impl.MongoConnectionConfig, MongoDBOutput" />
</configSections>
\end{code}

\subsection{Checkmarx Service Connection Configuration}\label{sec:connection}

\subsubsection{Checkmarx SAST Connection Configuration}

Configuring a connection to Checkmarx SAST requires the elements \texttt{CxSASTConnection} and \texttt{CxSASTCredentials}.

\begin{xml}{CxSASTConnection}{\expandsenv}{}
<CxSASTConnection
    URL=""
    mnoURL=""
    TimeoutSeconds="" 
    ValidateCertificates="true"
    RetryLoop=""
    />
\end{xml}

\begin{table}[h]
    \caption{CxSASTConnection Attributes}        
    \begin{tabularx}{\textwidth}{cccl}
        \toprule
        \textbf{Attribute} & \textbf{Default} & \textbf{Required} & \textbf{Description}\\
        \midrule
        \texttt{URL} & N/A & Yes & \makecell[l]{The URL to the SAST server.}\\
        \midrule
        \texttt{mnoURL} & N/A & No & \makecell[l]{The URL to the Management and Orchestration\\endpoint of the SAST server.}\\
        \midrule
        \texttt{TimeoutSeconds} & 300 & No & \makecell[l]{The number of seconds to wait until an\\API operation times out.}\\
        \midrule
        \texttt{ValidateCertificates} & True & No & \makecell[l]{Validate SSL certificates for\\API endpoints.}\\
        \midrule
        \texttt{RetryLoop} & 0 & No & \makecell[l]{The number of retries for an API operation\\after the operation times out.}\\
        \bottomrule
    \end{tabularx}
\end{table}

\begin{xml}{CxSASTCredentials}{\expandsenv\encrypts}{}
<CxSASTCredentials
    Username=""
    Password=""
    />
\end{xml}
    
\begin{table}[h]
    \caption{CxSASTCredentials Attributes}        
    \begin{tabularx}{\textwidth}{cccl}
        \toprule
        \textbf{Attribute} & \textbf{Default} & \textbf{Required} & \textbf{Description}\\
        \midrule
        \texttt{Username} & N/A & Yes & \makecell[l]{A username for a SAST application account.}\\
        \midrule
        \texttt{Password} & N/A & Yes & \makecell[l]{The password for the SAST application account.}\\
        \bottomrule
    \end{tabularx}
\end{table}

\subsubsection{Checkmarx SCA Connection Configuration}
Configuring a connection to Checkmarx SCA requires the elements \texttt{CxSCAConnection} and \texttt{CxSCACredentials}.

\begin{xml}{CxSCAConnection}{\expandsenv}{}
<CxSCAConnection
    URL=""
    LoginURL=""
    TimeoutSeconds="" 
    ValidateCertificates="true"
    RetryLoop=""
    />
\end{xml}

\begin{table}[h]
    \caption{CxSCAConnection Attributes}        
    \begin{tabularx}{\textwidth}{cccl}
        \toprule
        \textbf{Attribute} & \textbf{Default} & \textbf{Required} & \textbf{Description}\\
        \midrule
        \texttt{URL} & N/A & Yes & \makecell[l]{The URL to the SCA API.}\\
        \midrule
        \texttt{LoginURL} & N/A & No & \makecell[l]{The URL to the SCA access control endpoint.}\\
        \midrule
        \texttt{TimeoutSeconds} & 300 & No & \makecell[l]{The number of seconds to wait until an\\API operation times out.}\\
        \midrule
        \texttt{ValidateCertificates} & True & No & \makecell[l]{Validate SSL certificates for\\API endpoints.}\\
        \midrule
        \texttt{RetryLoop} & 0 & No & \makecell[l]{The number of retries for an API operation\\after the operation times out.}\\
        \bottomrule
    \end{tabularx}
\end{table}

\begin{xml}{CxSCACredentials}{\expandsenv\encrypts}{}
<CxSCACredentials
    Username=""
    Password=""
    Tenant=""
    />
\end{xml}
    
\begin{table}[h]
    \caption{CxSCACredentials Attributes}        
    \begin{tabularx}{\textwidth}{cccl}
        \toprule
        \textbf{Attribute} & \textbf{Default} & \textbf{Required} & \textbf{Description}\\
        \midrule
        \texttt{Username} & N/A & Yes & \makecell[l]{A username for an SCA application account.}\\
        \midrule
        \texttt{Password} & N/A & Yes & \makecell[l]{The password for the SCA application account.}\\
        \midrule
        \texttt{Tenant} & N/A & Yes & \makecell[l]{The name of the SCA tenant.}\\
        \bottomrule
    \end{tabularx}
\end{table}


\subsection{CxAnalytix Service and CLI Execution Configuration}\label{sec:runtime_config}

The \texttt{CxAnalytixService} element provides the runtime configuration for CxAnalytix.  The child element \texttt{EnabledTransformers}
is configured with the transformation logic modules to use when crawling Checkmarx services.

\begin{xml}{CxAnalytixService}{\expandsenv}{}
<CxAnalytixService
    InstanceId=""
    ConcurrentThreads=""
    StateDataStoragePath=""
    ProcessPeriodMinutes=""
    OutputModuleName=""
    SASTScanSummaryRecordName=""
    SASTScanDetailRecordName=""
    SCAScanSummaryRecordName=""
    SCAScanDetailRecordName=""
    ProjectInfoRecordName=""
    PolicyViolationsRecordName="">

    <EnabledTransformers>
        <Transformer Name="" />
    </EnabledTransformers>

</CxAnalytixService>
\end{xml}
        
\begin{table}[h]
    \caption{CxAnalytixService Attributes}        
    \begin{tabularx}{\textwidth}{cccl}
        \toprule
        \textbf{Attribute} & \textbf{Default} & \textbf{Required} & \textbf{Description}\\
        \midrule
        \texttt{InstanceId} & N/A & No & \makecell[l]{A static value added to each data record\\
        to indicate the CxAnalytix instance\\
        from which the record originated.}\\
        \midrule
        \texttt{ConcurrentThreads} & N/A & Yes & \makecell[l]{The number of reports that are processed\\
        concurrently.  This applies per \\
        transformation module, therefore using \\
        2 threads and 2 transformation modules\\
        yields 4 concurrent threads.}\\
        \midrule
        \texttt{StateDataStoragePath} & N/A & Yes & \makecell[l]{A path to a folder where the state data\\
        that is persisted between each scan is\\stored.}\\
        \midrule
        \texttt{ProcessPeriodMinutes} & N/A & Yes & \makecell[l]{The number of minutes between
        \\performing crawls for new scan\\
        results. Ignored by CxAnalytixCLI.}\\
        \midrule
        \texttt{OutputModuleName} & N/A & Yes & \makecell[l]{The name of the output module to use\\
        for data output.  The acceptable\\
        values can be found in\\
        the \hyperref[lst:outmodules]{Available Output Modules} list.}\\
        \midrule
        \texttt{SASTScanSummaryRecordName}\\
        \texttt{SASTScanDetailRecordName}\\
        \texttt{SCAScanSummaryRecordName}\\
        \texttt{SCAScanDetailRecordName}\\
        \texttt{ProjectInfoRecordName}\\
        \texttt{PolicyViolationsRecordName} & N/A & Yes & \makecell[tl]{The name of the corresponding\\
        record collection configured\\
        in the output.}\\
        \bottomrule
    \end{tabularx}
\end{table}


\paragraph{Available Output Modules}\label{lst:outmodules}
\begin{itemize}
    \item Log4Net
    \item AMQP
    \item MongoDB
\end{itemize}

\noindent\\The child element \texttt{EnabledTransformers} defines one or more transformer modules that crawl corresponding Checkmarx services.  The
attribute \texttt{Name} of the child element \texttt{Transformer} can be given one of the following values:\\

\begin{itemize}
    \item SAST
    \item SCA
\end{itemize}

\noindent\\One or more \texttt{Transformer} elements are required.  In the example below, both the SAST and SCA transformers are configured.
The selected services must have corresponding connection configurations as described in Section \ref{sec:connection}.

\begin{xml}{CxAnalytixService> \ \faArrowLeft \ <EnabledTransformers}{}{}
<CxAnalytixService ... >
    <EnabledTransformers>
        <Transformer Name="SAST" />
        <Transformer Name="SCA" />
    </EnabledTransformers>
</CxAnalytixService>
\end{xml}

\subsection{Limiting the Scope of Crawling by Filtering}

The optional \texttt{ProjectFilterRegex} configuration element can be used to limit the scope of the data crawl to only those scans matching Team or Project
name regular expressions.  The filtering is performed using a regular expression to evaluate Team and Project path such that the values of each must
match the provided regular expression.  If this \texttt{ProjectFilterRegex} is not included in the configuration file, all scans are crawled and exported.

\noindent\\The \texttt{Team} and \texttt{Project} attributes are optional.  Omitting one of the attributes or configuring the attribute with an empty
value indicates all values match.  Negative matching regular expressions also work; the typical application of this configuration is to limit crawling
to projects that are deployed to production.

\noindent\\In the example XML, the configuration crawls scans for projects meeting the following criteria:

\begin{itemize}
    \item The team does not contain the word "AppSec" anywhere in the team path.
    \item The project name contains the word "master".
\end{itemize}

\begin{xml}{ProjectFilterRegex}{}{}
<ProjectFilterRegex 
    Team="^((?!AppSec).)*\$" 
    Project="master"
    />
\end{xml}
    