\chapter{Forwarding Data to Splunk}\label{chap:splunk_config}

The Log4Net output is used to generate log files that are tailed and forwarded to Splunk via the Splunk Universal Forwarder.  This requires the Log4Net
output to be configured so that the Universal Forwarder can find the generated output files.  Please refer to Section \ref{sec:runtime_config} for details
about choosing the Log4Net output module.

\noindent\\The Log4Net configuration file should be modified only to change the output path of the generated data output files.  The generated data output files are different
than the application logging output files in that the data output files contain data to be used for analysis purposes. It may be desirable to also forward
the application log files to Splunk for monitoring and troubleshooting purposes.


\section{Configuring CxAnalytix for Splunk}\label{sec:splunk}

\noindent\\The \texttt{CxAnalytixService} configuration section in \texttt{cxanalytix.config} contains record name mapping attributes.  The XML example 
\hyperref[lst:record_map]{Record Map Configuration}
shows an example configuration with the record names mapped to file logger names shown configured in the example 
\hyperref[lst:record_loggers]{Log4Net Logger Configurations}.  The loggers
reference file appenders, as seen in the snippet \hyperref[lst:record_appenders]{Log4Net Record File Appenders}.  The appender configuration, 
by default, places all output in the \texttt{logs}
directory, which resolves to the current working directory set when a CxAnalytix process executes.  The location of the output files can be changed
by modifying the appender configuration in \texttt{cxanalytix.log4net}.



\begin{code}{Example Record Map Configuration}{\label{lst:record_map}}{}
<CxAnalytixService 
    ConcurrentThreads="2" 
    StateDataStoragePath="%CHECKMARX_STATE_PATH%"
    ProcessPeriodMinutes="120"
    OutputModuleName="log4net"
    SASTScanSummaryRecordName="RECORD_SAST_Scan_Summary"
    SASTScanDetailRecordName="RECORD_SAST_Scan_Detail"
    SCAScanSummaryRecordName="RECORD_SCA_Scan_Summary"
    SCAScanDetailRecordName="RECORD_SCA_Scan_Detail"
    ProjectInfoRecordName="RECORD_Project_Info"
    PolicyViolationsRecordName="RECORD_Policy_Violations">
    <EnabledTransformers>
        <Transformer Name="SAST" />
    </EnabledTransformers>
</CxAnalytixService>
\end{code}


\begin{code}{Log4Net Logger Configurations}{\label{lst:record_loggers}}{}
.. snip ..
<logger name="RECORD_SAST_Scan_Summary" aditivity="false">
    <level value="ALL" />
    <appender-ref ref="SAST_SS" />
</logger>
  .. snip ..
\end{code}


\begin{code}{Log4Net Record File Appenders}{\label{lst:record_appenders}}{}
.. snip ..
<appender name="SAST_SS" type="log4net.Appender.RollingFileAppender">
    <appendToFile value="true" />
    <maximumFileSize value="100MB" />
    <rollingStyle value="Composite" />
    <staticLogFileName value="false" />
    <countDirection value="1" />
    <file type="log4net.Util.PatternString" value="logs/sast_scan_summary" />
    <datePattern value="'.'yyyy_MM_dd'.log'" />
    <preserveLogFileNameExtension value="true" />

    <layout type="log4net.Layout.PatternLayout">
        <conversionPattern value="%message%newline" />
    </layout>
</appender>
.. snip ..
\end{code}

\section{Splunk Universal Forwarder Configuration}
The \ref{https://www.splunk.com/en_us/download/universal-forwarder.html}{Splunk Universal Forwarder} is used to send data to Splunk Enterprise or Splunk Cloud.  
Please refer to the Splunk website for information for details about installing and configuring the Universal Forwarder.


\subsection{Output File Tailing Configuration}

Assuming an installed forwarder is able to connect to the desired Splunk instance, 
create the \href{https://docs.splunk.com/Documentation/Splunk/latest/Admin/Inputsconf}{\texttt{inputs.conf}} file at the appropriate location 
(e.g. \texttt{/etc/apps/splunkclouduf/default/inputs.conf}).  The example \hyperref[lst:inputsconf]{inputs.conf configuration} below shows an example of 
an \texttt{inputs.conf} file with monitoring stanzas appropriate for each type of record. 


\begin{code}{inputs.conf Configuration}{\label{lst:inputsconf}}{}
[monitor://{path to logs}\CxAnalytixService...]
sourcetype=service

[monitor://{path to logs}\sast_scan_summary...]
sourcetype=sast_scan_summary

[monitor://{path to logs}\sast_scan_detail...]
sourcetype=sast_scan_detail

[monitor://{path to logs}\sast_project_info...]
sourcetype=sast_project_info

[monitor://{path to logs}\sast_policy_violations...]
sourcetype=sast_policy_violation

[monitor://{path to logs}\sca_scan_summary...]
sourcetype=sca_scan_summary

[monitor://{path to logs}\sca_scan_detail...]
sourcetype=sca_scan_detail

[monitor://{path to logs}\CxActivity_dbo_AuditTrail...]
sourcetype=cxactivity_audittrail

[monitor://{path to logs}\CxActivity_dbo_Audit_Scans...]
sourcetype=cxactivity_auditscans

[monitor://{path to logs}\CxActivity_dbo_Audit_Reports...]
sourcetype=cxactivity_auditreports

[monitor://{path to logs}\CxActivity_dbo_Audit_Queries...]
sourcetype=cxactivity_auditqueries

[monitor://{path to logs}\CxActivity_dbo_Audit_Projects...]
sourcetype=cxactivity_auditprojects

[monitor://{path to logs}\CxActivity_dbo_Audit_Presets...]
sourcetype=cxactivity_auditpresets

[monitor://{path to logs}\CxActivity_dbo_Audit_DataRetention...]
sourcetype=cxactivity_auditdataretention
\end{code}

\subsection{Configuring the Source Types}

The source types on the Splunk server need to be configured to appropriately parse JSON.  
This can be done using \href{https://docs.splunk.com/Documentation/Splunk/latest/Admin/Propsconf}{\texttt{props.conf}} (only available in Splunk Enterprise) 
or through the Splunk UI. A source type should be created the matches each record output source types as defined in 
\href{https://docs.splunk.com/Documentation/Splunk/latest/Admin/Inputsconf}{\texttt{inputs.conf}}.  The \hyperref[lst:sourcetypes]{configuration snippet below} shows an example
of a source type entry.  The source type configuration needs to be performed for each source type.


\begin{code}{Log4Net Record File Appenders}{\label{lst:sourcetypes}}{}
LINE_BREAKER=([\r\n]+)
KV_MODE=json
TRUNCATE=0
SHOULD_LINEMERGE=false
\end{code}

\noindent\\\textbf{Extracting Timestamps}\\

\noindent\\The source data contains timestamp fields that can be used as the timestamp Splunk uses when indexing the data.  Without specifying how to extract the
timestamp from each source type, the timestamp will default to the timestamp when the data was indexed.  This may work for current data, but data searches 
will also return historical data that is outside of the selected search time frame.

\noindent\\For the SAST Scan Summary, SAST Scan Detail, SCA Scan Summary, and SCA Scan Detail source types, this configuration option should be added:

\noindent\\\texttt{TIME\_PREFIX=\^.*ScanFinished".+?"}

\noindent\\For the SAST Project Info source type, this configuration option should be added:

\noindent\\\texttt{TIME\_PREFIX=\^.*LastCrawlDate".+?"}

\noindent\\For the SAST Policy Violation source type, this configuration option should be added:

\noindent\\\texttt{TIME\_PREFIX=\^.*ViolationOccurredDate".+?"}

\noindent\\For the Audit Trail source type, this configuration option should be added:

\noindent\\\texttt{TIME\_PREFIX=\^.*EndTime".+?"}

\noindent\\For the Audit\_Scans, Audit\_Reports, Audit\_Queries, Audit\_Projects, Audit\_Presets, Audit\_DataRetention, this configuration option should be added:

\noindent\\\texttt{TIME\_PREFIX=\^.*TimeStamp".+?"}

