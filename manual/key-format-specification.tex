\chapter{Key Format Specification}\label{chap:key_format_spec}

The Key Format Specification is a string composed of alphanumeric text, field specifiers, and field format specifiers.  It is used in various configuration
elements to dynamically generate a value containing record data that is added to data sent via the configured output method.

\noindent\\The syntax of a Key Format Specification is:

\noindent\\\texttt{\{field key[:format value]\}}

\noindent\\Where the \texttt{field key} element is the name of the field in the record from which to extract a value used when composing a string value.
The following example shows how to create a shard key from the data found in \texttt{ScanType}, \texttt{TeamName}, and \texttt{ProjectName}.

\noindent\\
\begin{xml}{Spec}{example with field expansion}{}
<Spec KeyName="pkey" CollectionName="SAST_Scan_Summary" 
    FormatSpec="SHARD-{ScanType}-{TeamName}-{ProjectName}"  />
\end{xml}


\noindent\\When the \texttt{field key} element is a dictionary type, a dotted notation may be used to reference a key in the dictionary
value referenced by \texttt{field key}.  The following example creates a shard key using the value of a custom field:

\begin{xml}{Spec}{example with field dotted notation}{}
<Spec KeyName="pkey" CollectionName="SAST_Scan_Summary" 
    FormatSpec="SHARD-{ScanType}-{TeamName}-{ProjectName}-{CustomFields.MyCustomField}"
    />
\end{xml}


\noindent\\The curly braces (\texttt{\{} and \texttt{\}}) can be embedded in the generated string by using a backslash (\texttt{\textbackslash}) to escape 
the curly brace.  The following example shows a shard key with contents surrounded by curly braces:

\begin{xml}{Spec}{example with escaped braces}{}
<Spec KeyName="pkey" CollectionName="SAST_Scan_Summary"
    FormatSpec="\{{ScanType}-{TeamName}-{ProjectName}\}"  />
\end{xml}

\section{Field Format Specifier}

The field format specifier follows the same convention of a .Net string format specifier.  The format string used depend on the data type:

\begin{itemize}
    \item\href{https://docs.microsoft.com/en-us/dotnet/standard/base-types/standard-date-and-time-format-strings}{Standard} and \href{https://docs.microsoft.com/en-us/dotnet/standard/base-types/custom-date-and-time-format-strings}{Custom} date and time format strings.
    \item\href{https://docs.microsoft.com/en-us/dotnet/standard/base-types/standard-timespan-format-strings}{Standard} and \href{https://docs.microsoft.com/en-us/dotnet/standard/base-types/custom-timespan-format-strings}{Custom} time span format strings.
    \item\href{https://docs.microsoft.com/en-us/dotnet/standard/base-types/standard-numeric-format-strings}{Standard} and \href{https://docs.microsoft.com/en-us/dotnet/standard/base-types/custom-numeric-format-strings}{Custom} numeric format strings.
\end{itemize}




